\iffalse
\let\negmedspace\undefined
\let\negthickspace\undefined
\documentclass[article]{IEEEtran}
       \def\inputGnumericTable{}                                 %%
\usepackage{cite}
\usepackage{amsmath,amssymb,amsfonts,amsthm}
\usepackage{algorithmic}
\usepackage{graphicx}
\usepackage{textcomp}
\usepackage{xcolor}
\usepackage{txfonts}
\usepackage{listings}
\usepackage{enumitem}
\usepackage{mathtools}
\usepackage{gensymb}
\usepackage[breaklinks=true]{hyperref}
\usepackage{tkz-euclide} % loads  TikZ and tkz-base
\usepackage{listings}
\renewcommand{\theenumi}{\Alph{enumi}}
%
%\usepackage{setspace}
%\usepackage{gensymb}
%\doublespacing
%\singlespacing

%\usepackage{graphicx}
%\usepackage{amssymb}
%\usepackage{relsize}
%\usepackage[cmex10]{amsmath}
%\usepackage{amsthm}
%\interdisplaylinepenalty=2500
%\savesymbol{iint}
%\usepackage{txfonts}
%\restoresymbol{TXF}{iint}
%\usepackage{wasysym}
%\usepackage{amsthm}
%\usepackage{iithtlc}
%\usepackage{mathrsfs}
%\usepackage{txfonts}
%\usepackage{stfloats}
%\usepackage{bm}
%\usepackage{cite}
%\usepackage{cases}
%\usepackage{subfig}
%\usepackage{xtab}
%\usepackage{longtable}
%\usepackage{multirow}
%\usepackage{algorithm}
%\usepackage{algpseudocode}
%\usepackage{enumitem}
%\usepackage{mathtools}
%\usepackage{tikz}
%\usepackage{circuitikz}
%\usepackage{verbatim}
%\usepackage{tfrupee}
%\usepackage{stmaryrd}
%\usetkzobj{all}
    \usepackage{color}                                            %%
    \usepackage{array}                                            %%
    \usepackage{longtable}                                        %%
    \usepackage{calc}                                             %%
    \usepackage{multirow}                                         %%
    \usepackage{hhline}                                           %%
    \usepackage{ifthen}                                           %%
 %optionally (for landscape tables embedded in another document): %%
    \usepackage{lscape}     
%\usepackage{multicol}
%\usepackage{chngcntr}
%\usepackage{enumerate}

%\usepackage{wasysym}
%\documentclass[conference]{IEEEtran}
%\IEEEoverridecommandlockouts
% The preceding line is only needed to identify funding in the first footnote. If that is unneeded, please comment it out.

\newtheorem{theorem}{Theorem}[section]
\newtheorem{problem}{Problem}
\newtheorem{proposition}{Proposition}[section]
\newtheorem{lemma}{Lemma}[section]
\newtheorem{corollary}[theorem]{Corollary}
\newtheorem{example}{Example}[section]
\newtheorem{definition}[problem]{Definition}
%\newtheorem{thm}{Theorem}[section] 
%\newtheorem{defn}[thm]{Definition}
%\newtheorem{algorithm}{Algorithm}[section]
%\newtheorem{cor}{Corollary}
\newcommand{\BEQA}{\begin{eqnarray}}
\newcommand{\EEQA}{\end{eqnarray}}
\newcommand{\define}{\stackrel{\triangle}{=}}
\theoremstyle{remark}
\newtheorem{rem}{Remark}

\begin{document}
\providecommand{\pr}[1]{\ensuremath{\Pr\left(#1\right)}}
\providecommand{\prt}[2]{\ensuremath{p_{#1}^{\left(#2\right)} }}        % own macro for this question
\providecommand{\qfunc}[1]{\ensuremath{Q\left(#1\right)}}
\providecommand{\sbrak}[1]{\ensuremath{{}\left[#1\right]}}
\providecommand{\lsbrak}[1]{\ensuremath{{}\left[#1\right.}}
\providecommand{\rsbrak}[1]{\ensuremath{{}\left.#1\right]}}
\providecommand{\brak}[1]{\ensuremath{\left(#1\right)}}
\providecommand{\lbrak}[1]{\ensuremath{\left(#1\right.}}
\providecommand{\rbrak}[1]{\ensuremath{\left.#1\right)}}
\providecommand{\cbrak}[1]{\ensuremath{\left\{#1\right\}}}
\providecommand{\lcbrak}[1]{\ensuremath{\left\{#1\right.}}
\providecommand{\rcbrak}[1]{\ensuremath{\left.#1\right\}}}
\newcommand{\sgn}{\mathop{\mathrm{sgn}}}
\providecommand{\abs}[1]{\left\vert#1\right\vert}
\providecommand{\res}[1]{\Res\displaylimits_{#1}} 
\providecommand{\norm}[1]{\left\lVert#1\right\rVert}
%\providecommand{\norm}[1]{\lVert#1\rVert}
\providecommand{\mtx}[1]{\mathbf{#1}}
\providecommand{\mean}[1]{E\left[ #1 \right]}
\providecommand{\cond}[2]{#1\middle|#2}
\providecommand{\fourier}{\overset{\mathcal{F}}{ \rightleftharpoons}}
\newenvironment{amatrix}[1]{%
  \left(\begin{array}{@{}*{#1}{c}|c@{}}
}{%
  \end{array}\right)
}
%\providecommand{\hilbert}{\overset{\mathcal{H}}{ \rightleftharpoons}}
%\providecommand{\system}{\overset{\mathcal{H}}{ \longleftrightarrow}}
	%\newcommand{\solution}[2]{\textbf{Solution:}{#1}}
\newcommand{\solution}{\noindent \textbf{Solution: }}
\newcommand{\cosec}{\,\text{cosec}\,}
\providecommand{\dec}[2]{\ensuremath{\overset{#1}{\underset{#2}{\gtrless}}}}
\newcommand{\myvec}[1]{\ensuremath{\begin{pmatrix}#1\end{pmatrix}}}
\newcommand{\mydet}[1]{\ensuremath{\begin{vmatrix}#1\end{vmatrix}}}
\newcommand{\myaugvec}[2]{\ensuremath{\begin{amatrix}{#1}#2\end{amatrix}}}
\providecommand{\rank}{\text{rank}}
\providecommand{\pr}[1]{\ensuremath{\Pr\left(#1\right)}}
\providecommand{\qfunc}[1]{\ensuremath{Q\left(#1\right)}}
	\newcommand*{\permcomb}[4][0mu]{{{}^{#3}\mkern#1#2_{#4}}}
\newcommand*{\perm}[1][-3mu]{\permcomb[#1]{P}}
\newcommand*{\comb}[1][-1mu]{\permcomb[#1]{C}}
\providecommand{\qfunc}[1]{\ensuremath{Q\left(#1\right)}}
\providecommand{\gauss}[2]{\mathcal{N}\ensuremath{\left(#1,#2\right)}}
\providecommand{\diff}[2]{\ensuremath{\frac{d{#1}}{d{#2}}}}
\providecommand{\myceil}[1]{\left \lceil #1 \right \rceil }
\newcommand\figref{Fig.~\ref}
\newcommand\tabref{Table~\ref}
\newcommand{\sinc}{\,\text{sinc}\,}
\newcommand{\rect}{\,\text{rect}\,}
%%
%	%\newcommand{\solution}[2]{\textbf{Solution:}{#1}}
%\newcommand{\solution}{\noindent \textbf{Solution: }}
%\newcommand{\cosec}{\,\text{cosec}\,}
%\numberwithin{equation}{section}
%\numberwithin{equation}{subsection}
%\numberwithin{problem}{section}
%\numberwithin{definition}{section}
%\makeatletter
%\@addtoreset{figure}{problem}
%\makeatother

%\let\StandardTheFigure\thefigure
\let\vec\mathbf

\bibliographystyle{IEEEtran}
\title{
%	\logo{
Assignment
%	}
}
\author{ Karthikeya hanu prakash kanithi (EE22BTECH11026)}
\maketitle
\parindent0px
\vspace{3cm}
Question : Consider communication over a memoryless binary symmetric channel using a
(7, 4) Hamming code. Each transmitted bit is received correctly with probability$(1 - \epsilon)$, and flipped with probability $\epsilon$. For each codeword transmission, the receiver
performs minimum Hamming distance decoding, and correctly decodes the message
bits if and only if the channel introduces at most one bit error.
\\For $\epsilon = 0.1$, the probability that a transmitted codeword is decoded correctly is
 \textunderscore\textunderscore\textunderscore\textunderscore\textunderscore\textunderscore (rounded off to two decimal places).
 (rounded off to two decimal places). 
\fi
\\ \solution 
Given that,
Let X be a random variable defined in the \autoref{tab:62/2022};
\begin{table}[h]
	\centering
	\input{gate/EC/2022/62/tables/Table2.tex}
	\caption{Random variable $X$ declaration}
        \label{tab:62/2022}
\end{table}
\\Then, $X \sim Bin(n,p)$ where 
\begin{align}
	n = 7 \quad p = \epsilon = 0.1 
	\label{eq:62/2022}
\end{align}
the pmf of X is given by
\begin{align}
    p_X(k) = \comb{7}{k} (\epsilon)^k (1 - \epsilon)^{7-k}
    \label{eq:62/2022/2}
\end{align}
the cdf of X is given by
\begin{align}
    F_X(k) = \sum_{i=0}^{k} \comb{7}{i} (\epsilon)^i (1 - \epsilon)^{7-i}
    \label{eq:62/2022/3}
\end{align}
From equation \eqref{eq:62/2022/3}, the probability of getting one or less error is given by 
\begin{align}
    F_X(1) &= \sum_{i=0}^{1} \comb{7}{i} (\epsilon)^i (1-\epsilon)^{7-i}\\
    &= \comb{7}{0} (\epsilon)^0 (1-\epsilon)^{7} + \comb{7}{1} (\epsilon)^1 (1-\epsilon)^{6}\\
    &= (1-\epsilon)^{7}+7(\epsilon)^1(1-\epsilon)^{6} \label{eq:62/2022/4}
\end{align}
From \eqref{eq:62/2022} and \eqref{eq:62/2022/4}, 
\begin{align}
    F_X(1) &= (1-0.1)^{7}+7(0.1)^1(1-0.1)^{6} \\
    &= 0.85
\end{align}
$\therefore$ the probability that a transmitted codeword is decoded correctly is 0.85.
\newpage
\textbf{Gaussian}\\
Let parameters be defined in the \autoref{tab:62/2022/2};
\begin{table}[h]
	\centering
	\input{gate/EC/2022/62/tables/Table3.tex}
	\caption{Parameters}
        \label{tab:62/2022/2}
\end{table}
\\Let Y is the Gaussian obtained by approximating binomial with parameters n,p then By Central limit theroem, 
\begin{align}
	X \approx Y \sim \mathcal{N}(np,npq)
\end{align}
We need to find 
\begin{align}
	\pr{Y \le 1} = F_Y(1)
\end{align}
After corrections to make the vlaues more accurate, we need to find
\begin{align}
	\pr{Y \le 1.33} = F_Y(1.33)
\end{align}
then CDF of $Y$ is:
\begin{align}
	F_Y(x)&=\pr{Y \le x} \\
	&=\pr{Y-\mu \le x-\mu} \\
	&=\pr{\frac{Y-\mu}{\sigma} \le \frac{x-\mu}{\sigma}} \label{eq:62/2022/5}
\end{align}
Since, 
\begin{align}
	\frac{Y-\mu}{\sigma} \sim \mathcal{N}(0,1)
\end{align}
Q function is defined
\begin{align}
	\qfunc{x} &= \pr{Y > x} \, \forall x \in Y \sim \mathcal{N}(0,1) \label{eq:62/2022/6}
\end{align}
From \eqref{eq:62/2022/5} and \eqref{eq:62/2022/6}, 
\begin{align}
	F_Y(x)&=1-\pr{\frac{Y-\mu}{\sigma} > \frac{x-\mu}{\sigma}}\\
	&= 
    \begin{cases}
        1-\qfunc{ \frac{x-\mu}{\sigma}}, &  x > \mu \\
        \qfunc{ \frac{\mu-x}{\sigma}} , &  x < \mu
    \end{cases} \label{eq:62/2022/7}
\end{align}
From \eqref{eq:62/2022/7} and \autoref{tab:62/2022/2},
\begin{align}
	F_Y(1.33)&=1-\qfunc{ \frac{1.33-0.7}{0.63}} \\
	&= 1-\qfunc{1}\\
	&= 0.8413
\end{align}
$\therefore$ the probability that a transmitted codeword is decoded correctly is 0.8413.
\\The Binomial CDF vs. Guassian CDF plot is given in fig\ref{fig:62/2022}
\begin{figure}[ht!]
    \centering
    \includegraphics[width=\columnwidth]{gate/EC/2022/62/figs/figure1.png}
    \caption{Binomial CDF vs. Guassian CDF}
    \label{fig:62/2022}
\end{figure}





















