\documentclass{article}
\usepackage{amsmath,amssymb,amsfonts,amsthm}

\usepackage{enumitem}
\providecommand{\pr}[1]{\ensuremath{\Pr\left(#1\right)}}
\usepackage{hyperref,xcolor}
\hypersetup{
    colorlinks,
    urlcolor={black}	%black!50!blue
}
\newcommand{\Mod}[1]{\ (\mathrm{mod}\ #1)}
\let\vec\mathbf

\def\inputGnumericTable{}
\usepackage{array}
\usepackage{longtable}
\usepackage{calc}
\usepackage{multirow}
\usepackage{hhline}
\usepackage{ifthen}

\providecommand{\cbrak}[1]{\ensuremath{\left\{#1\right\}}}
\newcommand{\Problem}{\noindent \textbf{Problem: }}
\newcommand{\solution}{\noindent \textbf{Solution: }}
\setlist[enumerate]{font=\small\bfseries}
\renewcommand\thefootnote{\textcolor{black}{\arabic{footnote}}}


\begin{document}


\title{Assignment: Probability}
\author{\Large T.Sai Raghavendra - FWC22087}
\date{}


\maketitle
\begin{enumerate}[label=16.\arabic{enumi}.\arabic{enumii}]%,ref=\thesection.\theenumi.\theenumi]
\numberwithin{equation}{enumi}
\numberwithin{table}{enumi}
%%%16.4.3
\setcounter{enumi}{3}
\setcounter{enumii}{3}

\item \footnote{Read question numbers as (CHAPTER NUMBER).(EXERCISE NUMBER).(QUESTION NUMBER)}A die has two faces each with number ‘1’, three faces each with number ‘2’ and one face with number ‘3’. If die is rolled once, determine
\begin{enumerate}
\item $\pr{2}$
\item $\pr{1 \text{ or } 3}$
\item $\pr{\text{not } 3}$
\end{enumerate}

\solution
	\begin{table}[h]
	\centering 
	
%%%%%%%%%%%%%%%%%%%%%%%%%%%%%%%%%%%%%%%%%%%%%%%%%%%%%%%%%%%%%%%%%%%%%%
%%                                                                  %%
%%  This is a LaTeX2e table fragment exported from Gnumeric.        %%
%%                                                                  %%
%%%%%%%%%%%%%%%%%%%%%%%%%%%%%%%%%%%%%%%%%%%%%%%%%%%%%%%%%%%%%%%%%%%%%%

\begin{center}
\begin{tabular}{|c|c|l|}
\hline
 Parameter&  Value &Description\\ \hline
 $X$ & $bin(n,p)$ & no of correct answers  that candidate gets by guessing\\\hline
$n$ &  5 & total no of questions\\ \hline
 $p$ &  $\frac{1}{3}$ & probability of getting correct answer by guessing\\\hline

\end{tabular}
\end{center}
	\caption{Variable Description.}
	\label{tables:16.4.3.2}
	\end{table}

\begin{enumerate}
\item \begin{align}
\pr{X_2} = \frac{1}{2}  %from \ref{tables:table2} 
\end{align}	
\item 
\begin{align}
\pr{X_1 + X_3}	&= \pr{X_1} + \pr{X_3} - \pr{X_1X_3}\\
				&= \frac{1}{3} + \frac{1}{6}   (\therefore \pr{X_1X_3} = 0)\\
				&= \frac{3}{6}\\
\pr{X_1 + X_3} 	&= \frac{1}{2}
\end{align}
\item 
\begin{align}
\pr{X_3^{\prime}} &= 1 - \pr{X_3}\\
			   &= 1 - \frac{1}{6}\\
			   &= \frac{5}{6}
\end{align}
\end{enumerate}
\end{enumerate}
\end{document}