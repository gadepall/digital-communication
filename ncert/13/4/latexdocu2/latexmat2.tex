\documentclass[12pt]{article}
\usepackage{graphicx}
\usepackage[table]{xcolor}
\usepackage{xfrac}
\usepackage{array}  
\usepackage{tabularx}
\usepackage{enumitem}
\usepackage{amsmath} 
\usepackage{mathtools}
\usepackage{gensymb}
       \usepackage[latin1]{inputenc}
       \usepackage{fullpage}
       \usepackage{color}
       \usepackage{longtable}
       \usepackage{calc}
       \usepackage{multirow}
       \usepackage{hhline}
       \usepackage{ifthen}
\def\inputGnumericTable{}

  % for having text in math mode
%Following 2 lines were added to remove the blank page at the beginning
\usepackage{atbegshi}% http://ctan.org/pkg/atbegshi
\AtBeginDocument{\AtBeginShipoutNext{\AtBeginShipoutDiscard}}
%

\begin{document}

\begin{center}
\title{\textbf{EXERCISE 13.4}}
\date{\vspace{-5ex}} %Not to print date automatically
\maketitle
\end{center}

\setcounter{page}{1}

\begin{enumerate}
\item State which of the following are not the probability distributions of a random 
variable. Give reasons for your answer
\renewcommand{\labelenumii}{\roman{enumii}}
\begin{enumerate}

\item \begin{table}[ht!]\centering
\input{tables/Book.tex}
\end{table}

\item \begin{table}[ht!]\centering
\input{tables/Book2.tex}
\end{table}

\item  \begin{table}[ht!]\centering
\input{tables/Book3.tex}	
\end{table}

\item  \begin{table}[ht!]\centering
\input{tables/Book5.tex}	
\end{table} 


\end{enumerate}
\item An urn contains 5 red and 2 black balls. Two balls are randomly drawn. Let X
represent the number of black balls. What are the possible values of X? Is X a
random variable ? 
\
\item Let X represent the difference between the number of heads and the number of
tails obtained when a coin is tossed 6 times. What are possible values of X?

\item Find the probability distribution of
\begin{enumerate}
\item number of heads in two tosses of a coin.
\item number of tails in the simultaneous tosses of three coins.
\item number of heads in four tosses of a coin.
\end{enumerate}

\item Find the probability distribution of the number of successes in two tosses of a die,
where a success is defined as
\begin{enumerate}
\item number greater than 4
\item six appears on at least one die
\end{enumerate}
 

\item From a lot of 30 bulbs which include 6 defectives, a sample of 4 bulbs is drawn
at random with replacement. Find the probability distribution of the number of
defective bulbs.

\item A coin is biased so that the head is 3 times as likely to occur as tail. If the coin is
tossed twice, find the probability distribution of number of tails.
\item A random variable X has the following probability distribution\\

Determine

\begin{enumerate}
\begin{table}[ht!]\centering
\input{tables/Book10.tex}
\end{table}
\item k
\item P$(X < 3)$
\item P$(X > 6)$
\item P$(0 < X < 3)$

\end{enumerate}

\item The random variable X has a probability distribution P(X) of the following form,
where k is some number :
\[P(x)=\begin{cases}
k, & \mbox{if}\; x= 0\\
2k, & \mbox{if}\; x= 1\\
3k, & \mbox{if}\; x= 2\\
0, & otherwise
\end{cases}\]
\begin{enumerate}
\item Determine the value of k.
\item Find P $(X < 2)$, P $(X \leq 2)$, P$(X \geq 2)$
\end{enumerate}
\item Find the mean number of heads in three tosses of a fair coin.

\item Two dice are thrown simultaneously. If X denotes the number of sixes, find the
expectation of X.

\item Two numbers are selected at random (without replacement) from the first six
positive integers. Let X denote the larger of the two numbers obtained. Find
E(X).

\item Let X denote the sum of the numbers obtained when two fair dice are rolled.
Find the variance and standard deviation of X.

\item A class has 15 students whose ages are 14, 17, 15, 14, 21, 17, 19, 20, 16, 18, 20,
17, 16, 19 and 20 years. One student is selected in such a manner that each has
the same chance of being chosen and the age X of the selected student is
recorded. What is the probability distribution of the random variable X? Find
mean, variance and standard deviation of X.

\item In a meeting, 70% of the members favour and 30% oppose a certain proposal.
A member is selected at random and we take X = 0 if he opposed, and X = 1 if
he is in favour. Find E(X) and Var (X).\\

Choose the correct answer in each of the following:



\item The mean of the numbers obtained on throwing a die having written 1 on three
faces, 2 on two faces and 5 on one face is
\begin{enumerate}
\item 1
\item 2
\item 5
\item $\frac{8}{3}$

\end{enumerate}

\item Suppose that two cards are drawn at random from a deck of cards. Let X be the
number of aces obtained. Then the value of E(X) is
\begin{enumerate}

\item $\frac{37}{221}$
\item $\frac{5}{13}$
\item $\frac{1}{13}$
\item $\frac{2}{13}$

\end{enumerate}

\end{enumerate}

\end{document}