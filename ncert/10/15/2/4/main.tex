\iffalse
\let\negmedspace\undefined
\let\negthickspace\undefined
\documentclass[journal,12pt,onecolumn]{IEEEtran}
%\documentclass[conference]{IEEEtran}
%\IEEEoverridecommandlockouts
% The preceding line is only needed to identify funding in the first footnote. If that is unneeded, please comment it out.
\usepackage{cite} %support for author-year citations 
\usepackage{amsmath,amssymb,amsfonts,amsthm}
\usepackage{algorithmic}
%\usepackage{graphicx}
\usepackage{textcomp} %text symbols, such as the degree symbol,currency,etc
\usepackage{xcolor} %colored text and background colors
\usepackage{txfonts} %e fonts are designed to be used in mathematical expressions, and include characters such as the Greek alphabet, mathematical symbols, and operators
\usepackage{multirow}
%\usepackage{enumitem} %customizing the appearance of lists.
\usepackage{mathtools} %exp,eqn,subscript,superscript
\usepackage{gensymb} %scientific symbols
\usepackage[breaklinks=true]{hyperref} %This package provides support for hyperlinks within a LaTeX document, including clickable references and URLs. 
%The breaklinks=true option tells LaTeX to break long URLs that would otherwise extend beyond the right margin of the page. This can help to improve the readability of the document, as it avoids the need to scroll horizontally to read long URLs.

\usepackage{tkz-euclide} % loads  TikZ and tkz-base
%tools for drawing geometric figures in a LaTeX document, with a focus on Euclidean geometry

\usepackage{listings} %The listings package can be used to display code from various programming languages, including C, C++, Java, Python, and many others. It also provides options to customize the appearance of the listings, such as changing the font, background color, line numbers, and more.
%
%\usepackage{setspace}  %commands to control the line spacing
%\usepackage{gensymb} 
%\doublespacing
%\singlespacing

%\usepackage{graphicx} %to include graphics (images) 
%\usepackage{amssymb} %additional mathematical symbols and fonts \subset, \supset, \in, \exists, \forall, \not\equiv, etc.
% example for real numberset \mathbb{R} 
%\usepackage{relsize} %provides a simple way to adjust the font size of math mode symbols and text in a flexible way. It provides the \mathlarger and \mathsmaller commands to adjust the size of math mode symbols and the \larger and \smaller commands to adjust the size of text
%\usepackage[cmex10]{amsmath}
%\usepackage{amsthm}
%\interdisplaylinepenalty=2500
%\savesymbol{iint}
%\usepackage{txfonts}
%\restoresymbol{TXF}{iint}
%\usepackage{wasysym}
%\usepackage{amsthm}
%\usepackage{iithtlc}
%\usepackage{mathrsfs}
%\usepackage{txfonts}
%\usepackage{stfloats}
%\usepackage{bm}
%\usepackage{cite}
%\usepackage{cases}
%\usepackage{subfig}
%\usepackage{xtab}
%\usepackage{longtable}
%\usepackage{multirow}
%\usepackage{algorithm}
%\usepackage{algpseudocode}
%\usepackage{enumitem}
%\usepackage{mathtools}
%\usepackage{tikz}
%\usepackage{circuitikz}
%\usepackage{verbatim}
%\usepackage{tfrupee}
%\usepackage{stmaryrd}
%\usetkzobj{all}
%    \usepackage{color}                                            %%
%    \usepackage{array}                                            %%
%    \usepackage{longtable}                                        %%
%    \usepackage{calc}                                             %%
%    \usepackage{multirow}                                         %%
%    \usepackage{hhline}                                           %%
%    \usepackage{ifthen}                                           %%
  %optionally (for landscape tables embedded in another document): %%
%    \usepackage{lscape}     
%\usepackage{multicol}
%\usepackage{chngcntr}
%\usepackage{enumerate}

%\usepackage{wasysym}
%\newcounter{MYtempeqncnt}
%\DeclareMathOperator*{\Res}{Res}
%\renewcommand{\baselinestretch}{2}
\renewcommand\thesection{\arabic{section}}
\renewcommand\thesubsection{\thesection.\arabic{subsection}}
\renewcommand\thesubsubsection{\thesubsection.\arabic{subsubsection}}

\renewcommand\thesectiondis{\arabic{section}}
\renewcommand\thesubsectiondis{\thesectiondis.\arabic{subsection}}
\renewcommand\thesubsubsectiondis{\thesubsectiondis.\arabic{subsubsection}}

% correct bad hyphenation here
\hyphenation{op-tical net-works semi-conduc-tor}
\def\inputGnumericTable{}                                 %%

\lstset{
%language=C,
frame=single, 
breaklines=true,
columns=fullflexible
}
%\lstset{
%language=tex,
%frame=single, 
%breaklines=true
%}

\begin{document}
%


\newtheorem{theorem}{Theorem}[section]
\newtheorem{problem}{Problem}
\newtheorem{proposition}{Proposition}[section]
\newtheorem{lemma}{Lemma}[section]
\newtheorem{corollary}[theorem]{Corollary}
\newtheorem{example}{Example}[section]
\newtheorem{definition}[problem]{Definition}
%\newtheorem{thm}{Theorem}[section] 
%\newtheorem{defn}[thm]{Definition}
%\newtheorem{algorithm}{Algorithm}[section]
%\newtheorem{cor}{Corollary}
\newcommand{\BEQA}{\begin{eqnarray}}
\newcommand{\EEQA}{\end{eqnarray}}
\newcommand{\define}{\stackrel{\triangle}{=}}

\bibliographystyle{IEEEtran}
%\bibliographystyle{ieeetr}


\providecommand{\mbf}{\mathbf}
\providecommand{\pr}[1]{\ensuremath{\Pr\left(#1\right)}}
\providecommand{\qfunc}[1]{\ensuremath{Q\left(#1\right)}}
\providecommand{\sbrak}[1]{\ensuremath{{}\left[#1\right]}}
\providecommand{\lsbrak}[1]{\ensuremath{{}\left[#1\right.}}
\providecommand{\rsbrak}[1]{\ensuremath{{}\left.#1\right]}}
\providecommand{\brak}[1]{\ensuremath{\left(#1\right)}}
\providecommand{\lbrak}[1]{\ensuremath{\left(#1\right.}}
\providecommand{\rbrak}[1]{\ensuremath{\left.#1\right)}}
\providecommand{\cbrak}[1]{\ensuremath{\left\{#1\right\}}}
\providecommand{\lcbrak}[1]{\ensuremath{\left\{#1\right.}}
\providecommand{\rcbrak}[1]{\ensuremath{\left.#1\right\}}}
\theoremstyle{remark}
\newtheorem{rem}{Remark}
\newcommand{\sgn}{\mathop{\mathrm{sgn}}}
\providecommand{\abs}[1]{\left\vert#1\right\vert}
\providecommand{\res}[1]{\Res\displaylimits_{#1}} 
\providecommand{\norm}[1]{\left\lVert#1\right\rVert}
%\providecommand{\norm}[1]{\lVert#1\rVert}
\providecommand{\mtx}[1]{\mathbf{#1}}
\providecommand{\mean}[1]{E\left[ #1 \right]}
\providecommand{\fourier}{\overset{\mathcal{F}}{ \rightleftharpoons}}
%\providecommand{\hilbert}{\overset{\mathcal{H}}{ \rightleftharpoons}}
\providecommand{\system}{\overset{\mathcal{H}}{ \longleftrightarrow}}
	%\newcommand{\solution}[2]{\textbf{Solution:}{#1}}
\newcommand{\solution}{\noindent \textbf{Solution: }}
\newcommand{\cosec}{\,\text{cosec}\,}
\providecommand{\dec}[2]{\ensuremath{\overset{#1}{\underset{#2}{\gtrless}}}}
\newcommand{\myvec}[1]{\ensuremath{\begin{pmatrix}#1\end{pmatrix}}}
\newcommand{\mydet}[1]{\ensuremath{\begin{vmatrix}#1\end{vmatrix}}}
%\numberwithin{equation}{section}
%\numberwithin{equation}{subsection}
%\numberwithin{problem}{section}
%\numberwithin{definition}{section}
%\makeatletter
%\@addtoreset{figure}{problem}
%\makeatother

%\let\StandardTheFigure\thefigure
\let\vec\mathbf

\vspace{3cm}

\title{
Assignment 2\\AI1110 : Probability and Random Variables
}
\author{Tumarada Padmaja\\CS22BTECH11059}

% make the title area
\maketitle
%\newpage
\bigskip
\textbf{Question:10.15.2.4:} 
A box contains 12 balls out of which x are black. If one ball is drawn at random from the box, what is the probability that it will be a black ball? If 6 more black balls are put in the box, the probability of drawing a black ball is now double of what it was before. Find x.
\\
\\
\textbf{Solution:} 
\fi
\begin{table}[htbp]
\centering
%%%%%%%%%%%%%%%%%%%%%%%%%%%%%%%%%%%%%%%%%%%%%%%%%%%%%%%%%%%%%%%%%%%%%%
%%                                                                  %%
%%  This is a LaTeX2e table fragment exported from Gnumeric.        %%
%%                                                                  %%
%%%%%%%%%%%%%%%%%%%%%%%%%%%%%%%%%%%%%%%%%%%%%%%%%%%%%%%%%%%%%%%%%%%%%%
\begin{tabular}{|c|c|c|c|c|}
\hline
Random Variable	&Sample space	&Value	&Event	&Probability\\
\hline
\multirow{2}{*}{$X_1$}	&\multirow{2}{*}{12}	&0	&not choosing black ball	&12-x/12\\
\cline{3-5}		&	&1	&choosing black ball	&x/12\\
\hline
\multirow{2}{*}{$X_2$}	&\multirow{2}{*}{18}	&0	&not choosing black ball	&12-x/18\\
\cline{3-5}		&	&1	&choosing black ball	&x+6/18\\
\hline
\end{tabular}

\caption{ }
\label{tab:10/15/2/4/1}
\end{table}
From
\tabref{tab:10/15/2/4/1},
\begin{align}
\pr{X_1=1} = \frac{x}{12} 
\end{align}
Since
\begin{align}
\pr{X_2=1} &= 2\pr{X_1=1},\\
	\frac{x+6}{18} &=2\brak{\frac{x}{12}}\\
\implies
x &=3
\end{align}


