\iffalse
\documentclass[a4paper,10pt,two column]{article}
\usepackage[english]{babel}
\usepackage[letterpaper,top=2cm,bottom=2cm,left=3cm,right=3cm,marginparwidth=1.75cm]{geometry}
\newcommand*{\rom}[1]

\usepackage{multicol}
\usepackage{amsmath}
\usepackage{graphicx}
\usepackage{array}
\usepackage{blindtext}
\usepackage[utf8]{inputenc}
\usepackage{watermark}
\usepackage[colorlinks=true, allcolors=blue]{hyperref}
\usepackage{listings}
\usepackage{float}
\providecommand{\cbrak}[1]{\ensuremath{\left\{#1\right\}}}
\providecommand{\sbrak}[1]{\ensuremath{{}\left[#1\right]}}

\providecommand{\pr}[1]{\ensuremath{\Pr\left(#1\right)}}
\usepackage{amsmath,amssymb,amsfonts,amsthm}
\title{PROBABILITY ASSIGNMENT}
\author{MOHAMMAD IMRAN}
\thiswatermark{\centering \put(-50,-105){\includegraphics[scale=0.5]{iith.png}} }
\begin{document}
\maketitle
%\tableofcontents
\bigskip


\section{\textbf{Problem }}
Two dice are thrown simultaneously, if x denotes number os sixes, find the expectation of x


\section{\textbf{Solution }}
Consider each trial results in success (i.e getting sixes on dice) or failure (i.e not getting sixes on dice) represented by 1 and 0 respectively.
\\

Let $X_i \in \cbrak{1,2,3,4,5,6} , i = 1,2$ be the random variables representing the outcome for each die.Assuming the dice to be fair,the probability mass function(pmf) is expressed as
\\
\begin{align}
 p_{X_i}(k) = \Pr(X_i=k) =
\begin{cases}
\frac{1}{6} & 1 \leq k \leq 6 \\
0 & otherwise
\end{cases} 
\end{align}
$p$ and $q = (1 - p)$ are the probability of success and failure respectively.
\begin{align}
& p = P_{X_i}(1) = \frac{1}{6}&               \label{eq:1}
\\            
& q = 1 - p = P_{X_i}(0) = \frac{5}{6}&       \label{eq:2}
\end{align}



The generating function (or z-transform) of $\pr{X_i =k}$  is defined as 
\\
\begin{align}
M_{X_i}(z) = E\sbrak{z^{X_i}} &= \sum_{k=-\infty}^{\infty} P_{X_i}(k)z^{-k}&
\\
E\sbrak{z^{X_1+X_2+...+X_n}}&= E\sbrak{z^{X_1}z^{X_2}....z^{X_n}}& \nonumber
\\
&= E\sbrak{z^{X_1}}E\sbrak{z^{X_2}}....E\sbrak{z^{X_n}}&
\end{align} 

($\because X_1,X_2,...,X_n$are independent and identically distributed)
\\
$\therefore$For a simulatenous throw of two die,
\begin{align}
E\sbrak{z^{X_i}}& = \sum_{k=0}^{2} P_{X_i}(k) z^{-k}&
\\
&= (q+pz^{-1})^{2}&
\end{align}
$\Rightarrow M_{X_i}(z)$ for n number of dice is
\begin{align}
 M_{X_i}(z)&= (q + pz^{-1})^n&
\end{align}

Expectation of x or mean is defined as the 1st moment of $M_{X_i}(z^{-1})$ at z=1 i.e,
\begin{align}
\mu&= \frac{dM_{X_i}(z^{-1})}{dz}|_{z=1}&
\\
&= \frac{d(q + pz)^n}{dz}|_{z=1}&
\\
&=np(q + pz)^{n-1}|_{z=1}&
\\
&= np(q + p)^{n-1}&
\\
\therefore \text{Mean} &= np& {}\because(p+q=1)
\\
&= 2\times\frac{1}{6}= 0.333&
\end{align}
\\

The Expectation of x  is $0.33$
\\


\end{document}
