\documentclass{article}
\usepackage{amssymb,amsfonts,amsthm,amsmath}
\usepackage{enumitem}
\providecommand{\pr}[1]{\ensuremath{\Pr\left(#1\right)}}
\providecommand{\cbrak}[1]{\ensuremath{\left\{#1\right\}}}
\newcommand{\solution}{\noindent \textbf{Solution: }}
\newcommand*{\permcomb}[4][0mu]{{{}^{#3}\mkern#1#2_{#4}}}
\newcommand*{\perm}[1][-3mu]{\permcomb[#1]{P}}
\newcommand*{\comb}[1][-1mu]{\permcomb[#1]{C}}
\setlist[enumerate]{font=\small\bfseries}
\renewcommand\thefootnote{\textcolor{black}{\arabic{footnote}}}
\def\inputGnumericTable{}
\usepackage[latin1]{inputenc}
\usepackage{fullpage}
\usepackage{color}
\usepackage{array}
\usepackage{longtable}
\usepackage{calc}
\usepackage{multirow}
\usepackage{hhline}
\usepackage{ifthen}
\usepackage{graphicx}
\graphicspath{ {./images/} }
\providecommand{\pr}[1]{\ensuremath{\Pr\left(#1\right)}}

\begin{document}

\title{PROBABILITY}
\author{\Large T SIVA PARVATHI - FWC22089}
\date{}

\maketitle
\begin{enumerate}[label=13.\arabic{enumi}.\arabic{enumii}]%,ref=\thesection.\theenumi.\theenumi]
\numberwithin{equation}{enumi}
%\numberwithin{table}{enumi}
\setcounter{enumi}{3}
\setcounter{enumii}{5}

\item \footnote{Read question numbers as (CHAPTER NUMBER).(EXERCISE NUMBER).(QUESTION NUMBER)}
Find the probability distribution of the number of successes in two tosses of a die, where a success is defined as
\begin{enumerate}
\item number greater than 4
\item six appears on at least one die
\end{enumerate}

\solution
Given that a die tossed two times,
\begin{table}[h]\centering
	\input{dtable.tex}
	 \caption{Variable Description}
	 \label{tab:13.4.5}
\end{table}

Refer \ref{tab:13.4.5} for numericals,
\begin{enumerate}
\item number greater than 4

$X_1$ and $X_2$ are independent events, so the desired outcome is
\begin{align}
X&=X_1+X_2
\end{align}
In $n$ Bernoulli trials with $k$ success and $(n - k)$ failures, the probablity of $k$ success in $n$-Bernoulli trials can be given as,
\begin{align}
p_{X_i}(k)   &= 
\begin{cases}
\comb{n}{k}{p}^{k}{q}^{n-k} & 0 \le k \le n\\
0 & \text{otherwise}                
\end{cases}
\end{align}
Probability distribution of getting number greater than 4,
\begin{align}
p_X(k)   &= \comb{n}{k}{p_1}^{k}{q_1}^{n-k}, 0 \le k \le 2
\end{align}

\begin{figure}
\centering
\includegraphics[width=0.6\columnwidth]{bfig1.pdf}
\caption{Stem plot for the distribution $\pr{X>4}$}
\label{fig:Plot}
\end{figure}

\item six appears on at least one die

$X_1$ and $X_2$ are independent events, so the desired outcome is
\begin{align}
X&=X_1+X_2
\end{align}
Probability distribution of getting six on atleast one die is,
\begin{align}
p_X(k)   &= \comb{n}{k}{p_2}^{k}{q_2}^{n-k}, 0 \le k \le 2
\end{align}

\begin{figure}
\centering
\includegraphics[width=0.6\columnwidth]{bfig2.pdf}
\caption{Stem plot for the distribution $\pr{X=6}$}
\label{fig:Plot}
\end{figure}
\end{enumerate}
\end{enumerate}
\end{document}