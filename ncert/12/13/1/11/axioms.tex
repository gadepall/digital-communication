\iffalse
%\documentclass[class=article, crop=false]{standalone}
\documentclass{article}
\usepackage{amssymb,amsfonts,amsthm,amsmath}
\usepackage{enumitem}
\usepackage{hyperref,xcolor}
\hypersetup{
    colorlinks,
    urlcolor={black}	%black!50!blue
}
\providecommand{\pr}[1]{\ensuremath{\Pr\left(#1\right)}}
%\documentclass{article}	% working
\def\inputGnumericTable{}
\usepackage[latin1]{inputenc}
\usepackage{fullpage}
\usepackage{color}
\usepackage{array}
\usepackage{longtable}
\usepackage{calc}
\usepackage{multirow}
\usepackage{hhline}
\usepackage{ifthen}
\providecommand{\cbrak}[1]{\ensuremath{\left\{#1\right\}}}
\newcommand{\solution}{\noindent \textbf{Solution: }}
\providecommand{\pr}[1]{\ensuremath{\Pr\left(#1\right)}}
%\newcommand{\varsol}{\noindent \textbf{Aliter: }}
\newcommand*{\permcomb}[4][0mu]{{{}^{#3}\mkern#1#2_{#4}}}
%\newcommand*{\perm}[1][-3mu]{\permcomb[#1]{P}}
\newcommand*{\comb}[1][-1mu]{\permcomb[#1]{C}}
\setlist[enumerate]{font=\small\bfseries}
\renewcommand\thefootnote{\textcolor{black}{\arabic{footnote}}}

\begin{document}

\title{PROBABILITY}
\author{\Large HARI VENKATESWARLU - FWC22058}
\date{}

\maketitle

\begin{enumerate}[label=13.\arabic{enumi}.\arabic{enumii}]%,ref=\thesection.\theenumi.\theenumi]
\numberwithin{equation}{enumi}
\numberwithin{table}{enumi}
\setcounter{enumi}{0}
\setcounter{enumii}{11}

\item \footnote{Read question numbers as (CHAPTER NUMBER).(EXERCISE NUMBER).(QUESTION NUMBER)} 
	\solution\\
	\fi
	The given information is summarised in Table
	\ref{table:}.
	\begin{table}[h]\centering
	
%%%%%%%%%%%%%%%%%%%%%%%%%%%%%%%%%%%%%%%%%%%%%%%%%%%%%%%%%%%%%%%%%%%%%%
%%                                                                  %%
%%  This is a LaTeX2e table fragment exported from Gnumeric.        %%
%%                                                                  %%
%%%%%%%%%%%%%%%%%%%%%%%%%%%%%%%%%%%%%%%%%%%%%%%%%%%%%%%%%%%%%%%%%%%%%%

\begin{center}
\begin{tabular}{|c|c|l|}
\hline
 Parameter&  Value &Description\\ \hline
 $X$ & $bin(n,p)$ & no of correct answers  that candidate gets by guessing\\\hline
$n$ &  5 & total no of questions\\ \hline
 $p$ &  $\frac{1}{3}$ & probability of getting correct answer by guessing\\\hline

\end{tabular}
\end{center}
	\caption{Probability of Events.}
	\label{table:}
\end{table}
\begin{enumerate}
\item 
\begin{align}
\pr{E \mid F} &= \frac{\pr{EF}}{\pr{F}}\\
&=\frac{\frac{1}{6}}{\frac{1}{3}}\\
&=\frac{1}{2}\\
\pr{F \mid E} &= \frac{\pr{FE}}{\pr{E}}\\
&= \frac{\frac{1}{6}}{\frac{1}{2}}\\
&=\frac{1}{3}
\end{align}
\item 
\begin{align}
\pr{E \mid G} &= \frac{\pr{EG}}{\pr{G}}\\
&=\frac{\frac{1}{3}}{\frac{2}{3}}\\
&=\frac{1}{2}\\
\pr{G \mid E} &= \frac{\pr{GE}}{\pr{G}}\\
&=\frac{\frac{1}{3}}{\frac{1}{2}}\\
&=\frac{2}{3}
\end{align}
\item 
\begin{align}
\pr{{E+F} \mid G} &= \frac{\pr{{(E+F)}G}}{\pr{G}}\\
&=\frac{\pr{{EG+F}G}}{\pr{G}}\\
&=\frac{\pr{EG}+\pr{FG}-\pr{EFG}}{\pr{G}}\\
&=\frac{3}{4}\\
\pr{{EF} \mid G} &= \frac{\pr{EFG}}{\pr{G}}\\
&=\frac{1}{4}
\end{align}
\end{enumerate}
