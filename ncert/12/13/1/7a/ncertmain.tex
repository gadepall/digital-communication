\iffalse
\documentclass[journal,11pt,twocolumn]{IEEEtran}
\usepackage{setspace}
\usepackage{gensymb}
\singlespacing
\usepackage[cmex10]{amsmath}
\usepackage{amsthm}
\usepackage{mathrsfs}
\usepackage{txfonts}
\usepackage{stfloats}
\usepackage{bm}
\usepackage{cite}
\usepackage{cases}
\usepackage{subfig}
\usepackage{longtable}
\usepackage{multirow}
\usepackage{enumitem}
\usepackage{mathtools}
\usepackage{tikz}
\usepackage{circuitikz}
\usepackage{verbatim}
\usepackage[breaklinks=true]{hyperref}
\usepackage{tkz-euclide} % loads  TikZ and tkz-base
\usepackage{listings}
\usepackage{color}    
\usepackage{array}    
\usepackage{longtable}
\usepackage{calc}     
\usepackage{multirow} 
\usepackage{hhline}   
\usepackage{ifthen}   
\usepackage{lscape}     
\usepackage{chngcntr}
\DeclareMathOperator*{\Res}{Res}
\renewcommand\thesection{\arabic{section}}
\renewcommand\thesubsection{\thesection.\arabic{subsection}}
\renewcommand\thesubsubsection{\thesubsection.\arabic{subsubsection}}

\renewcommand\thesectiondis{\arabic{section}}
\renewcommand\thesubsectiondis{\thesectiondis.\arabic{subsection}}
\renewcommand\thesubsubsectiondis{\thesubsectiondis.\arabic{subsubsection}}
\renewcommand\thetable{\arabic{table}}
% correct bad hyphenation here
\hyphenation{op-tical net-works semi-conduc-tor}
\def\inputGnumericTable{}                                 %%

\lstset{
%language=C,
frame=single, 
breaklines=true,
columns=fullflexible
}
%\lstset{
%language=tex,
%frame=single, 
%breaklines=true
%}

\begin{document}
\parindent 0px
\newtheorem{theorem}{Theorem}[section]
\newtheorem{problem}{Problem}
\newtheorem{proposition}{Proposition}[section]
\newtheorem{lemma}{Lemma}[section]
\newtheorem{corollary}[theorem]{Corollary}
\newtheorem{example}{Example}[section]
\newtheorem{definition}[problem]{Definition}
\newcommand{\BEQA}{\begin{eqnarray}}
\newcommand{\EEQA}{\end{eqnarray}}
\newcommand{\define}{\stackrel{\triangle}{=}}
\bibliographystyle{IEEEtran}
\providecommand{\mbf}{\mathbf}
\providecommand{\pr}[1]{\ensuremath{\Pr\left(#1\right)}}
\providecommand{\qfunc}[1]{\ensuremath{Q\left(#1\right)}}
\providecommand{\sbrak}[1]{\ensuremath{{}\left[#1\right]}}
\providecommand{\lsbrak}[1]{\ensuremath{{}\left[#1\right.}}
\providecommand{\rsbrak}[1]{\ensuremath{{}\left.#1\right]}}
\providecommand{\brak}[1]{\ensuremath{\left(#1\right)}}
\providecommand{\lbrak}[1]{\ensuremath{\left(#1\right.}}
\providecommand{\rbrak}[1]{\ensuremath{\left.#1\right)}}
\providecommand{\cbrak}[1]{\ensuremath{\left\{#1\right\}}}
\providecommand{\lcbrak}[1]{\ensuremath{\left\{#1\right.}}
\providecommand{\rcbrak}[1]{\ensuremath{\left.#1\right\}}}
\theoremstyle{remark}
\newtheorem{rem}{Remark}
\newcommand{\sgn}{\mathop{\mathrm{sgn}}}
\providecommand{\abs}[1]{\left\vert#1\right\vert}
\providecommand{\res}[1]{\Res\displaylimits_{#1}} 
\providecommand{\norm}[1]{\left\lVert#1\right\rVert}
\providecommand{\mtx}[1]{\mathbf{#1}}
\providecommand{\mean}[1]{E\left[ #1 \right]}
\providecommand{\fourier}{\overset{\mathcal{F}}{ \rightleftharpoons}}
\providecommand{\system}[1]{\overset{\mathcal{#1}}{ \longleftrightarrow}}
\newcommand{\solution}{\noindent \textbf{Solution: }}
\newcommand{\cosec}{\,\text{cosec}\,}
\providecommand{\dec}[2]{\ensuremath{\overset{#1}{\underset{#2}{\gtrless}}}}
\newcommand{\myvec}[1]{\ensuremath{\begin{pmatrix}#1\end{pmatrix}}}
\newcommand{\mydet}[1]{\ensuremath{\begin{vmatrix}#1\end{vmatrix}}}
\let\vec\mathbf
\def\putbox#1#2#3{\makebox[0in][l]{\makebox[#1][l]{}\raisebox{\baselineskip}[0in][0in]{\raisebox{#2}[0in][0in]{#3}}}}
     \def\rightbox#1{\makebox[0in][r]{#1}}
     \def\centbox#1{\makebox[0in]{#1}}
     \def\topbox#1{\raisebox{-\baselineskip}[0in][0in]{#1}}
     \def\midbox#1{\raisebox{-0.5\baselineskip}[0in][0in]{#1}}

\vspace{3cm}
\title{Probability Assignment}
\author{EE22BTECH11022-Garikapati Sai Harshith}
\maketitle
\bigskip

Question: Find $P(E|F)$ for
\begin{enumerate}
\item$E$: tail appears on one coin.\\
   $F$: head appears on one coin.
\item$E$: no tail appears.\\
    $F$: no head appears.
\end{enumerate}
\fi
\solution 
The random variables $X_1$ and $X_2$ are shown in Table \ref{tab:rvdef}.
\begin{table}[!ht]
\centering

%%%%%%%%%%%%%%%%%%%%%%%%%%%%%%%%%%%%%%%%%%%%%%%%%%%%%%%%%%%%%%%%%%%%%%
%%                                                                  %%
%%  This is a LaTeX2e table fragment exported from Gnumeric.        %%
%%                                                                  %%
%%%%%%%%%%%%%%%%%%%%%%%%%%%%%%%%%%%%%%%%%%%%%%%%%%%%%%%%%%%%%%%%%%%%%%

\begin{center}
\begin{tabular}{|c|c|l|}
\hline
 Parameter&  Value &Description\\ \hline
 $X$ & $bin(n,p)$ & no of correct answers  that candidate gets by guessing\\\hline
$n$ &  5 & total no of questions\\ \hline
 $p$ &  $\frac{1}{3}$ & probability of getting correct answer by guessing\\\hline

\end{tabular}
\end{center}
\caption{Definition of $X_1$ and $X_2$.}
\label{tab:rvdef}
\end{table}\\
    Since the coins are fair.
\begin{align}
P_{X_1X_2}(k,m)=\frac{1}{4}
\label{eq:12.13.1.7,1}
\end{align}
In \eqref{eq:12.13.1.7,1}, $k,m \in \cbrak{0,1}$. So, total four different k,m combinations.
    \begin{enumerate}
        \item $E$: Here one coin is tail and other is head. We are required to find $\pr{X_1+X_2=1}$. Thus, from \eqref{eq:12.13.1.7,1}.
        \begin{align}
        \pr{E}&=\pr{X_1+X_2=1}\\
        &=\pr{X_1=0,\ X_2=1}+\pr{X_1=1,\ X_2=0}\\
        &=\frac{1}{2}
        \end{align}
        $F$: Here one coin is head and other is tail. We are required to find $\pr{X_1+X_2=1}$. Thus, from \eqref{eq:12.13.1.7,1}.
        \begin{align}
        \pr{F}&=\pr{X_1+X_2=1}\\
        &=\pr{X_1=0,\ X_2=1}+\pr{X_1=1,\ X_2=0}\\
        &=\frac{1}{2}
        \end{align}
  	$EF$: Here one coin is head and other is tail. We are required to find $\pr{X_1+X_2=1}$. Thus, from \eqref{eq:12.13.1.7,1}.
\begin{align}
        \pr{EF}&=\pr{X_1+X_2=1}\\
        &=\pr{X_1=0,\ X_2=1}+\pr{X_1=1,\ X_2=0}\\
        &=\frac{1}{2}\\
        \pr{E|F}&=\frac{\pr{EF}}{\pr{F}}\\
        &=\frac{\frac{1}{2}}{\frac{1}{2}}\\
        &=1
\end{align}
\item $E$: no tail appears.We are required to find $\pr{X_1\neq0,X_2\neq0}$. Thus, from \eqref{eq:12.13.1.7,1}.
\begin{align}
\pr{E}&=\pr{X_1\neq0,X_2\neq0}\\
&=\pr{X_1=1,\ X_2=1}\\
&=\frac{1}{4}
        \end{align}
        $F$: no head appears.We are required to find $\pr{X_1\neq1,X_2\neq1}$. Thus, from \eqref{eq:12.13.1.7,1}.
        \begin{align}
        \pr{F}&=\pr{X_1\neq1,X_2\neq1}\\
        &=\pr{X_1=0,\ X_2=0}\\
        &=\frac{1}{4}
\end{align}
$EF$: coins should show neither head nor tail. From Table \ref{tab:rvdef}, we have coins showing head or tail. So, this is an impossible event
\begin{align}
\pr{EF}&=0\\
\pr{E|F}&=\frac{\pr{EF}}{\pr{F}}\\
&=\frac{0}{\frac{1}{4}}\\
&=0
\end{align}
        
        
     
    \end{enumerate}
