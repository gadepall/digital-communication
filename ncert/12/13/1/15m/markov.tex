\iffalse
\documentclass[journal,12pt,two column]{IEEEtran}
%\usepackage{setspace}
\usepackage{listings}
\usepackage{amssymb}
\usepackage[cmex10]{amsmath}
\usepackage{amsthm}
\usepackage[export]{adjustbox}
\usepackage{bm}
\def\inputGnumericTable{} 

\usepackage[latin1]{inputenc}                                 
\usepackage{color}                                            
\usepackage{array} 
\usepackage{longtable} 
\usepackage{calc}                                             
\usepackage{multirow}                                         
\usepackage{hhline}                                           
\usepackage{ifthen}  
\usepackage{mathtools}
\usepackage{tikz}
\usepackage{listings}
\usepackage{color}                                            %%
\usepackage{array}                                            %%
\usepackage{caption} 
\usepackage{graphicx}

\title{A1110 Assignment 3 \\ 12.13.1.15}
\author{K.SaiTeja \\ AI22BTECH11014}
\providecommand{\pr}[1]{\ensuremath{\Pr\left(#1\right)}}
\providecommand{\qfunc}[1]{\ensuremath{Q\left(#1\right)}}
\providecommand{\sbrak}[1]{\ensuremath{{}\left[#1\right]}}
\providecommand{\lsbrak}[1]{\ensuremath{{}\left[#1\right.}}
\providecommand{\rsbrak}[1]{\ensuremath{{}\left.#1\right]}}
\providecommand{\brak}[1]{\ensuremath{\left(#1\right)}}
\providecommand{\lbrak}[1]{\ensuremath{\left(#1\right.}}
\providecommand{\rbrak}[1]{\ensuremath{\left.#1\right)}}
\providecommand{\cbrak}[1]{\ensuremath{\left\{#1\right\}}}
\providecommand{\lcbrak}[1]{\ensuremath{\left\{#1\right.}}
\providecommand{\rcbrak}[1]{\ensuremath{\left.#1\right\}}}
\newcommand*{\permcomb}[4][0mu]{{{}^{#3}\mkern#1#2_{#4}}}
\newcommand*{\perm}[1][-3mu]{\permcomb[#1]{P}}
\newcommand*{\comb}[1][-1mu]{\permcomb[#1]{C}}

\renewcommand{\thetable}{\arabic{table}} 
\newcommand{\question}{\noindent \textbf{Question: }}	
\newcommand{\solution}{\noindent \textbf{Solution: }}
\providecommand{\mbf}{\mathbf}
\providecommand{\pr}[1]{\ensuremath{\Pr\left(#1\right)}}
\providecommand{\prt}[2]{\ensuremath{P_{#1}^{\left(#2\right)} }}        % own macro for this question
\providecommand{\qfunc}[1]{\ensuremath{Q\left(#1\right)}}
\providecommand{\sbrak}[1]{\ensuremath{{}\left[#1\right]}}      % []
\providecommand{\lsbrak}[1]{\ensuremath{{}\left[#1\right.}}
\providecommand{\rsbrak}[1]{\ensuremath{{}\left.#1\right]}}
\providecommand{\brak}[1]{\ensuremath{\left(#1\right)}}         % ()
\providecommand{\lbrak}[1]{\ensuremath{\left(#1\right.}}
\providecommand{\rbrak}[1]{\ensuremath{\left.#1\right)}}
\providecommand{\cbrak}[1]{\ensuremath{\left\{#1\right\}}}      % {}
\providecommand{\lcbrak}[1]{\ensuremath{\left\{#1\right.}}
\providecommand{\rcbrak}[1]{\ensuremath{\left.#1\right\}}}
\theoremstyle{remark}
\newtheorem{rem}{Remark}
\newcommand{\sgn}{\mathop{\mathrm{sgn}}}
\providecommand{\abs}[1]{\ensuremath{\left\vert#1\right\vert}}
\providecommand{\res}[1]{\Res\displaylimits_{#1}} 
\providecommand{\norm}[1]{\ensuremath{\left\lVert#1\right\rVert}}
%\providecommand{\norm}[1]{\lVert#1\rVert}
\providecommand{\mtx}[1]{\mathbf{#1}}
\providecommand{\mean}[1]{\ensuremath{E\left[ #1 \right]}}
\providecommand{\fourier}{\overset{\mathcal{F}}{ \rightleftharpoons}}
%\providecommand{\hilbert}{\overset{\mathcal{H}}{ \rightleftharpoons}}
\providecommand{\system}{\overset{\mathcal{H}}{ \longleftrightarrow}}
	%\newcommand{\solution}[2]{\textbf{Solution:}{#1}}
\newcommand{\cosec}{\,\text{cosec}\,}
\providecommand{\dec}[2]{\ensuremath{\overset{#1}{\underset{#2}{\gtrless}}}}
\newcommand{\myvec}[1]{\ensuremath{\begin{pmatrix}#1\end{pmatrix}}}
\newcommand{\mydet}[1]{\ensuremath{\begin{vmatrix}#1\end{vmatrix}}}
%
%not used because document is short:
%\numberwithin{equation}{section}
%\numberwithin{figure}{section}
%\numberwithin{table}{section}
%\numberwithin{equation}{section}
%\numberwithin{problem}{section}
%\numberwithin{definition}{section}
\makeatletter
\@addtoreset{figure}{problem}
\makeatother

\let\StandardTheFigure\thefigure
\let\vec\mathbf
%\renewcommand{\thefigure}{\theproblem.\arabic{figure}}
    %\renewcommand{\thefigure}{\theproblem}
%\setlist[enumerate,1]{before=\renewcommand\theequation{\theenumi.\arabic{equation}}
%\counterwithin{equation}{enumi}
%\renewcommand{\theequation}{\arabic{subsection}.\arabic{equation}}

\def\putbox#1#2#3{\makebox[0in][l]{\makebox[#1][l]{}\raisebox{\baselineskip}[0in][0in]{\raisebox{#2}[0in][0in]{#3}}}}
     \def\rightbox#1{\makebox[0in][r]{#1}}
     \def\centbox#1{\makebox[0in]{#1}}
     \def\topbox#1{\raisebox{-\baselineskip}[0in][0in]{#1}}
     \def\midbox#1{\raisebox{-0.5\baselineskip}[0in][0in]{#1}}
\vspace{3cm}

%\renewcommand{\thefigure}{\theenumi}
%\renewcommand{\thetable}{\theenumi}
%\renewcommand{\theequation}{\theenumi}

\begin{document}
\maketitle
\question
\solution\tableofcontents
\fi
\begin{enumerate}
	\item  See 
        \tabref{tab:ncert/12/13/1/15m/States}	
	and 
        \figref{fig:ncert/12/13/1/15m/ markov_chain}.
    \begin{table}[ht!]
        \centering
    	%%%%%%%%%%%%%%%%%%%%%%%%%%%%%%%%%%%%%%%%%%%%%%%%%%%%%%%%%%%%%%%%%%%%%%
%%                                                                  %%
%%  This is a LaTeX2e table fragment exported from Gnumeric.        %%
%%                                                                  %%
%%%%%%%%%%%%%%%%%%%%%%%%%%%%%%%%%%%%%%%%%%%%%%%%%%%%%%%%%%%%%%%%%%%%%%

\begin{center}
\begin{tabular}{|c|c|c|}
\hline
\textbf{RV}& \textbf{Values} & \textbf{Description} \\ \hline
$X$		   & 	$\{0,1\}$	&  1st draw - 0: Red, 1: Black\\ \hline
$Y$ 		   & 	$\{0,1\}$	&  2nd draw - 0: Red, 1: Black\\ \hline
\end{tabular}
\end{center}

        \caption{States in Markov Chain}
        \label{tab:ncert/12/13/1/15m/States}	
    \end{table}
\begin{figure}[!ht]
        \centering
        \input{ncert/12/13/1/15m/figs/graph1.tex}  
        \caption{Graph of Markov Chain}
        \label{fig:ncert/12/13/1/15m/ markov_chain}
\end{figure}
\item 
    The state vector is,   
    \begin{align}
        \vec{Q}_n &= \myvec{\prt{0}{n}\\ 
        			        \prt{1}{n} \\
        			        \prt{2}{n} \\
        			        \prt{3}{n} 
        			        }        
    \end{align}    
    The probabilities after one step in time are
    \begin{align}
       \prt{0}{n+1} &= \frac{2}{6} \times \prt{0}{n}  \\
       \prt{1}{n+1} &= \frac{4}{6} \times \prt{0}{n}  \\
       \prt{2}{n+1} &= \frac{1}{2} \times \prt{1}{n} + 1 \times \prt{2}{n}
\\
       \prt{3}{n+1} &= \frac{1}{2} \times \prt{1}{n} + 1 \times \prt{3}{n} 
    \end{align}
   \item  
    The previous equations can be summarized as
    \begin{align}
        \vec{Q}_{n+1} &= \vec{P}\vec{Q}_n 
        \label{eq:ncert/12/13/1/15m/transtition}        
    \end{align}
		Where $\vec{P}$ is the transition probability matrix. Its elements are values of $p_{i|j}$
    \begin{align}
        \vec{P}= \myvec{%%%%%%%%%%%%%%%%%%%%%%%%%%%%%%%%%%%%%%%%%%%%%%%%%%%%%%%%%%%%%%%%%%%%%%
%%                                                                  %%
%%  This is a LaTeX2e table fragment exported from Gnumeric.        %%
%%                                                                  %%
%%%%%%%%%%%%%%%%%%%%%%%%%%%%%%%%%%%%%%%%%%%%%%%%%%%%%%%%%%%%%%%%%%%%%%

\begin{center}
\begin{tabular}{|c|c|}
\hline
\textbf{Event}& \textbf{Probability} \\ \hline
$\pr{A = 1}$ & 	$\frac{4}{5}$ \\ \hline
$\pr{X = 1}$ & 	$\frac{1}{2}$ \\ \hline
$\pr{X = 1 \mid A= 1}$ &  $\frac{1}{2}$ \\ \hline
\end{tabular}
\end{center}
} 
    \end{align}
\item 
     The given condition is that \lq3 occurs at least once\rq. Let the first occurrence of 3 be the initial state $ \vec{Q}_0$.
    \begin{align}
        \vec{Q}_0 &= \myvec{ 1 \\ 0 \\ 0 \\ 0 } 
    \end{align}
    Using \eqref{eq:ncert/12/13/1/15m/transtition}, further states can be generated.
    \begin{align}
        \vec{Q}_1 &= \vec{P} \vec{Q}_0
            = \myvec{\frac{2}{6} \\[4pt] \frac{4}{6}  \\[4pt] 0 \\ 0}\\
        \vec{Q}_2 &= \vec{P} \vec{Q}_1 = \vec{P}^{2} \vec{Q}_0 
            = \myvec{\frac{4}{9}  \\[4pt] \frac{8}{9}  \\[4pt] \frac{5}{24}\\[4pt] \frac{5}{12}} \\   
        \vdots \\
        \vec{Q}_n &= \vec{P}^{n} \vec{Q}_0
    \end{align}
    \item 
 Now to find the eigen values, 
 \begin{align}
\mydet{\vec{P}-\lambda \vec{I}} &= 0  
\\
 \implies \myvec{\input{ncert/12/13/1/15m/tables/table4.txt}} &= 0
 \\
	 \implies \lambda\brak{\frac{2}{6} - \lambda}\brak{1 - \lambda^2} &= 0
	 \\
	 \text{or, }\lambda &= \frac{2}{6}, 0 , 1 , 1 
 \end{align}
 The corresponding eigenvectors are
\begin{enumerate}
\item $\lambda = \frac{2}{6}$
\begin{align}
\vec{X} &= \myvec{\frac{-2}{3}\\[4pt] \frac{-4}{3}\\[4pt] 1\\[4pt] 1}
\end{align}
\item $\lambda = 0$
\begin{align}
\vec{X} &= \myvec{0 \\[2pt] -2\\[2pt] 1\\ 1}
\end{align}
\item $\lambda = 1$
\begin{align}
\vec{X} &= \myvec{0 \\ 0 \\ 1 \\ 0},\, 
\myvec{0 \\ 0 \\ 0 \\ 1}\\
\end{align}
\end{enumerate}
resulting in the 
eigenvector matrix
\begin{align}
\vec{S} &= \myvec{\input{ncert/12/13/1/15m/tables/table5.txt}}
\end{align}
Thus, 
\begin{align}
	\vec{P} &= \vec{S}\vec{D}\vec{S}^{-1}
\end{align}
Where $\vec{D}$ is eigenvalue matrix
\begin{align}
\vec{D} &= \myvec{\input{ncert/12/13/1/15m/tables/table6.txt}}
\end{align}
\item
\begin{align}
\vec{P}^{n} &= (\vec{S}\vec{D}\vec{S}^{-1})(\vec{S}\vec{D}\vec{S}^{-1}) \dots (\vec{S}\vec{D}\vec{S}^{-1})\\
\implies  &= \vec{S}\vec{D}^{n}\vec{S}^{-1}\\
\implies \lim_{n \to \infty}\vec{P}^{n} &= \lim_{n \to \infty}\vec{S}\vec{D}^{n}\vec{S}^{-1}
\end{align}
	From 
        \eqref{eq:ncert/12/13/1/15m/transtition},
\begin{align}
\vec{Q}_{n} &= \vec{P}^{n}\vec{Q}_{0}
\end{align}
and
Now,
\begin{align}
\vec{\lim_{n \to \infty}\vec{D}^{n}} &= \myvec{\input{ncert/12/13/1/15m/tables/table7.txt}}\\ 
\implies \vec{\lim_{n \to \infty}\vec{S}\vec{D}^{n}\vec{S}^{-1}} &= \myvec{\input{ncert/12/13/1/15m/tables/table7.txt}} = \vec{\vec{P}^{n}}\\
\implies \vec{Q}_n &= \myvec{\input{ncert/12/13/1/15m/tables/table7.txt}}\vec{Q}_0\\
\implies \vec{Q}_n &= \myvec{0 \\ 0 \\ 0 \\ 0} 
\end{align}
\item 
    Probability of the coin showing tails, given that at least one die shows a 3,
    \begin{align}
       \lim_{n \to \infty} \prt{3}{n} = 0
    \end{align}
   \end{enumerate} 
