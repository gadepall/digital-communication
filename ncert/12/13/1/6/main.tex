\iffalse
\documentclass[journal,12pt,twocolumn]{IEEEtran}
\usepackage{romannum}
\usepackage{float}
\usepackage{setspace}
\usepackage{gensymb}
\singlespacing
\usepackage[cmex10]{amsmath}
\usepackage{amsthm}
\usepackage{mathrsfs}
\usepackage{txfonts}
\usepackage{stfloats}
\usepackage{bm}
\usepackage{cite}
\usepackage{cases}
\usepackage{subfig}
\usepackage{longtable}
\usepackage{multirow}
\usepackage{enumitem}
\usepackage{mathtools}
\usepackage{steinmetz}
\usepackage{tikz}
\usepackage{circuitikz}
\usepackage{verbatim}
\usepackage{tfrupee}
\usepackage[breaklinks=true]{hyperref}
\usepackage{tkz-euclide}
\usetikzlibrary{calc,math}
\usepackage{listings}
    \usepackage{color}                                            %%
    \usepackage{array}                                            %%
    \usepackage{longtable}                                        %%
    \usepackage{calc}                                             %%
    \usepackage{multirow}                                         %%
    \usepackage{hhline}                                           %%
    \usepackage{ifthen}                                           %%
  %optionally (for landscape tables embedded in another document): %%
    \usepackage{lscape}     
\usepackage{multicol}
\usepackage{chngcntr}
\DeclareMathOperator*{\Res}{Res}
\renewcommand\thesection{\arabic{section}}
\renewcommand\thesubsection{\thesection.\arabic{subsection}}
\renewcommand\thesubsubsection{\thesubsection.\arabic{subsubsection}}

\renewcommand\thesectiondis{\arabic{section}}
\renewcommand\thesubsectiondis{\thesectiondis.\arabic{subsection}}
\renewcommand\thesubsubsectiondis{\thesubsectiondis.\arabic{subsubsection}}

% correct bad hyphenation here
\hyphenation{op-tical net-works semi-conduc-tor}
\def\inputGnumericTable{}                                 %%

\lstset{
frame=single, 
breaklines=true,
columns=fullflexible
}

\begin{document}


\newtheorem{theorem}{Theorem}[section]
\newtheorem{problem}{Problem}
\newtheorem{proposition}{Proposition}[section]
\newtheorem{lemma}{Lemma}[section]
\newtheorem{corollary}[theorem]{Corollary}
\newtheorem{example}{Example}[section]
\newtheorem{definition}[problem]{Definition}
\newcommand{\BEQA}{\begin{eqnarray}}
\newcommand{\EEQA}{\end{eqnarray}}
\newcommand{\define}{\stackrel{\triangle}{=}}

\bibliographystyle{IEEEtran}
\providecommand{\mbf}{\mathbf}
\providecommand{\pr}[1]{\ensuremath{\Pr\left(#1\right)}}
\providecommand{\qfunc}[1]{\ensuremath{Q\left(#1\right)}}
\providecommand{\sbrak}[1]{\ensuremath{{}\left[#1\right]}}
\providecommand{\lsbrak}[1]{\ensuremath{{}\left[#1\right.}}
\providecommand{\rsbrak}[1]{\ensuremath{{}\left.#1\right]}}
\providecommand{\brak}[1]{\ensuremath{\left(#1\right)}}
\providecommand{\lbrak}[1]{\ensuremath{\left(#1\right.}}
\providecommand{\rbrak}[1]{\ensuremath{\left.#1\right)}}
\providecommand{\cbrak}[1]{\ensuremath{\left\{#1\right\}}}
\providecommand{\lcbrak}[1]{\ensuremath{\left\{#1\right.}}
\providecommand{\rcbrak}[1]{\ensuremath{\left.#1\right\}}}
\theoremstyle{remark}
\newtheorem{rem}{Remark}
\newcommand{\sgn}{\mathop{\mathrm{sgn}}}
\providecommand{\abs}[1]{\left\vert#1\right\vert}
\providecommand{\res}[1]{\Res\displaylimits_{#1}} 
\providecommand{\norm}[1]{\left\lVert#1\right\rVert}
\providecommand{\mtx}[1]{\mathbf{#1}}
\providecommand{\mean}[1]{E\left[ #1 \right]}
\providecommand{\fourier}{\overset{\mathcal{F}}{ \rightleftharpoons}}
\providecommand{\system}{\overset{\mathcal{H}}{ \longleftrightarrow}}
\newcommand{\solution}{\noindent \textbf{Solution: }}
\newcommand{\cosec}{\,\text{cosec}\,}
\providecommand{\dec}[2]{\ensuremath{\overset{#1}{\underset{#2}{\gtrless}}}}
\newcommand{\myvec}[1]{\ensuremath{\begin{pmatrix}#1\end{pmatrix}}}
\newcommand{\mydet}[1]{\ensuremath{\begin{vmatrix}#1\end{vmatrix}}}
\numberwithin{equation}{subsection}
\makeatletter
\@addtoreset{figure}{problem}
\makeatother

\let\StandardTheFigure\thefigure
\let\vec\mathbf
\renewcommand{\thefigure}{\theproblem}



\def\putbox#1#2#3{\makebox[0in][l]{\makebox[#1][l]{}\raisebox{\baselineskip}[0in][0in]{\raisebox{#2}[0in][0in]{#3}}}}
     \def\rightbox#1{\makebox[0in][r]{#1}}
     \def\centbox#1{\makebox[0in]{#1}}
     \def\topbox#1{\raisebox{-\baselineskip}[0in][0in]{#1}}
     \def\midbox#1{\raisebox{-0.5\baselineskip}[0in][0in]{#1}}

\vspace{3cm}


\title{Assignment 1}
\author{Jaswanth Chowdary Madala}





% make the title area
\maketitle

\newpage

%\tableofcontents

\bigskip

\renewcommand{\thefigure}{\theenumi}
\renewcommand{\thetable}{\theenumi}



\begin{enumerate}
\textbf{Solution:} 
		\fi
		Consider the random variables $X_1, X_2, X_3,X$, which denotes the first, second, third toss and number of heads in the 3 tosses respectively as described in table \ref{tab:ncert/12/13/1/6/1}.
\begin{table}[h]
\centering
%%%%%%%%%%%%%%%%%%%%%%%%%%%%%%%%%%%%%%%%%%%%%%%%%%%%%%%%%%%%%%%%%%%%%%
%%                                                                  %%
%%  This is a LaTeX2e table fragment exported from Gnumeric.        %%
%%                                                                  %%
%%%%%%%%%%%%%%%%%%%%%%%%%%%%%%%%%%%%%%%%%%%%%%%%%%%%%%%%%%%%%%%%%%%%%%

\begin{center}
\begin{tabular}{|c|c|c|}
\hline
\textbf{RV}& \textbf{Values} & \textbf{Description} \\ \hline
$X$		   & 	$\{0,1,2,3\}$	&   Number of heads in 3 tosses\\ \hline
$X_1$		   & 	$\{0,1\}$	&   0: Heads , 1: Tails\\ \hline
$X_2$ 		   & 	$\{0,1\}$	&	0: Heads , 1: Tails\\ \hline
$X_3$		   & 	$\{0,1\}$	&   0: Heads , 1: Tails\\ \hline
\end{tabular}
\end{center}

\caption{Random variables $X_1, X_2, X_3, X$}
\label{tab:ncert/12/13/1/6/1}
\end{table}

The random variable $X$ follows binomial distribution
\begin{align}
X = X_1 + X_2 + X_3
\end{align}
The PMF of the random variable $X$ is given by,
\begin{align}
P_{X}\brak{n} = {^{N}C_{n}} \, {p}^{n} \brak{1-p}^{N-n}
\end{align}
Here we have
\begin{align}
N = 3, \, p = \frac{1}{2}
\end{align}

The CDF of the random variable $X$ is given by,
\begin{align}
F_{X}\brak{n} = 	\pr{X \le n} \label{eq:ncert/12/13/1/6/2}
&= \sum_{i = 0}^{n} {^{N}C_{i}}\, {p}^{i} \brak{1-p}^{N-i}
\end{align}

\begin{enumerate}
\item  The events $E, F$ can be described by the RV as
\begin{align}
E &: X_3 = 0\\
F &: X_1 + X_2 = 0 
\end{align}
$Y$ is another random variable which represents the number of heads in first two tosses.
\begin{align}
Y = X_1 + X_2
\end{align} 
The PMF of the random variable $Y$ is given by,
\begin{align}
P_{Y}\brak{n} = {^{N}C_{n}} \, {p}^{n} \brak{1-p}^{N-n}
\end{align}
Here we have
\begin{align}
N = 2, \, p = \frac{1}{2}
\end{align}
The event $EF$ can be expressed as, 
\begin{align}
X_3 = 0 \, \cap \, X_1 + X_2 &= 0  \\
\triangleq  X_1 + X_2 + X_3 &= 0\\
\implies X &= 0
\end{align}
The required probability is given by,
\begin{align}
\pr{X_3 = 0 \mid Y = 0}
\end{align}
\begin{align}
&= \frac{\pr{X = 0}}{\pr{Y = 0}}\\
&= \frac{1}{2}
\end{align}

\item  The events $E, F, F^{\prime}$ can be described by the RV as
\begin{align}
E &: X \leq 1 \label{eq:ncert/12/13/1/6/3}\\
F &: X \geq 1 \label{eq:ncert/12/13/1/6/4}\\
F^{\prime} &: X = 0 
\end{align}
The required probability is given by,
\begin{align}
&= \frac{\pr{EF}}{1- \pr{F^{\prime}}}
\end{align}
The event $EF$ can be expressed as, 
\begin{align}
X \leq 1 \, &\cap \, X \geq 1\\
\implies X &= 1
\end{align}
Hence, the required probability is given by,
\begin{align}
&= \frac{\pr{X = 1}}{1 - \pr{X = 0}}  \label{eq:ncert/12/13/1/6/5}\\
&= \frac{\frac{3}{8}}{1-\frac{1}{8}}\\
&= \frac{3}{7}
\end{align}


\item  For the events $E, F$, their complements are
$E^{\prime}$ : all 3 tails, $F^{\prime}$ : zero tails. The events $E^{\prime}, F^{\prime}$ can be described by the RV as
\begin{align}
E^{\prime} &: X = 3\\
F^{\prime} &: X = 0 
\end{align}
By using property of conditional probability we have,
\begin{align}
\pr{E \mid F} &= \frac{\pr{EF}}{\pr{F}} \label{eq:ncert/12/13/1/6/1}\\
&= \frac{1 - \pr{E^{\prime} + F^{\prime}}}{\pr{F}}
\end{align}
The required probability is given by,
\begin{align}
&= \frac{1 - \pr{X = 0 + X = 3}}{1 - \pr{X = 0}}\\
&= \frac{1 - \brak{\pr{X = 0} + \pr{X = 3}- \pr{\phi}}}{1 - \pr{X = 0}}\\
&= \frac{1 - \brak{\frac{1}{8} + \frac{1}{8} - 0}}{1-\frac{1}{8}}\\
&= \frac{6}{7}
\end{align}

\end{enumerate}



