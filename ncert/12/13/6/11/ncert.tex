\iffalse
\let\negmedspace\undefined
\let\negthickspace\undefined
\documentclass[journal,12pt,twocolumn]{IEEEtran}
\usepackage{cite}
\usepackage{amsmath,amssymb,amsfonts,amsthm}
\usepackage{algorithmic}
\usepackage{graphicx}
\usepackage{textcomp}
\usepackage{xcolor}
\usepackage{txfonts}
\usepackage{listings}
\usepackage{enumitem}
\usepackage{mathtools}
\usepackage{gensymb}
\usepackage{comment}
\usepackage[breaklinks=true]{hyperref}
\usepackage{tkz-euclide} 
\usepackage{listings}
\usepackage{gvv}                                        
\def\inputGnumericTable{}                                 
\usepackage[latin1]{inputenc}                                
\usepackage{color}                                            
\usepackage{array}                                            
\usepackage{longtable}                                       
\usepackage{calc}                                             
\usepackage{multirow}                                         
\usepackage{hhline}                                           
\usepackage{ifthen}                                           
\usepackage{lscape}

\newtheorem{theorem}{Theorem}[section]
\newtheorem{problem}{Problem}
\newtheorem{proposition}{Proposition}[section]
\newtheorem{lemma}{Lemma}[section]
\newtheorem{corollary}[theorem]{Corollary}
\newtheorem{example}{Example}[section]
\newtheorem{definition}[problem]{Definition}
\newcommand{\BEQA}{\begin{eqnarray}}
\newcommand{\EEQA}{\end{eqnarray}}
\newcommand{\define}{\stackrel{\triangle}{=}}
\theoremstyle{remark}
\newtheorem{rem}{Remark}
\begin{document}

\bibliographystyle{IEEEtran}
\vspace{3cm}

\title{ASSIGNMENT}
\author{EE22BTECH11016-Chinthalapudi Yashwanth$^{*}$% <-this % stops a space
}
\maketitle
\newpage
\bigskip
\renewcommand{\thefigure}{\theenumi}
\renewcommand{\thetable}{\theenumi}

\textbf{Question 12.13.6.11}
In a game, a man wins a rupee for a six and loses a rupee for any other number
when a fair die is thrown. The man decided to throw a die thrice but to quit as
and when he gets a six. Find the expected value of the amount he wins / loses.\\
\fi
\solution
Random variables defined as
\begin{table}[!ht]
	
%%%%%%%%%%%%%%%%%%%%%%%%%%%%%%%%%%%%%%%%%%%%%%%%%%%%%%%%%%%%%%%%%%%%%%
%%                                                                  %%
%%  This is a LaTeX2e table fragment exported from Gnumeric.        %%
%%                                                                  %%
%%%%%%%%%%%%%%%%%%%%%%%%%%%%%%%%%%%%%%%%%%%%%%%%%%%%%%%%%%%%%%%%%%%%%%

\begin{center}
\begin{tabular}{|c|c|l|}
\hline
 Parameter&  Value &Description\\ \hline
 $X$ & $bin(n,p)$ & no of correct answers  that candidate gets by guessing\\\hline
$n$ &  5 & total no of questions\\ \hline
 $p$ &  $\frac{1}{3}$ & probability of getting correct answer by guessing\\\hline

\end{tabular}
\end{center}
\end{table}\\
$k$ be roll on which 6 appeared.Let $m_k$ be money recieved till 6 appeared.
\begin{align}
p_{X}(k)&=\begin{cases}
            \brak{\frac{5}{6}}^{k-1} \frac{1}{6} & \text{if } k \in \{1,2,3\}\\
            \brak{\frac{5}{6}}^{3} & \text{if } k =0 \\
            0 & \text{otherwise}
        \end{cases}\\
m_k &=\begin{cases}
            (-1)(k-1)+1 & \text{if } k \in \{1,2,3\}\\
            -3 & \text{if } k =0 
        \end{cases}
\end{align}
Calculating the expected value
\begin{align}
\text{Expected value} & = \sum^{3}_{k=0} m_k p_{X}(k)\\
&=\brak{1 \times \frac{1}{6}} + \brak{0 \times \frac{5}{36}} + \brak{-1 \times \frac{25}{216}}\notag \\
    &+ \brak{-3 \times \frac{125}{216}} \\
    &= \frac{1}{6} - 0 + \brak{-\frac{25}{216}}- \frac{375}{216} \\
    &= \frac{36}{216} - 0 + \brak{-\frac{25}{216}} - \frac{375}{216} \\
    &= \frac{36 - 0 - 25 - 375}{216} \\
    &= -\frac{364}{216} \\
    &\approx -1.685
\end{align}
