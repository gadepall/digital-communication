\iffalse
\documentclass[journal,12pt,twocolumn]{IEEEtran}
\usepackage{setspace}
\usepackage{gensymb}
\singlespacing
\usepackage[cmex10]{amsmath}
\usepackage{amsthm}
\usepackage{mathrsfs}
\usepackage{txfonts}
\usepackage{stfloats}
\usepackage{bm}
\usepackage{cite}
\usepackage{cases}
\usepackage{subfig}
\usepackage{longtable}
\usepackage{multirow}
\usepackage{enumitem}
\usepackage{mathtools}
\usepackage{tikz}
\usepackage{circuitikz}
\usepackage{verbatim}
\usepackage[breaklinks=true]{hyperref}
\usepackage{tkz-euclide} % loads  TikZ and tkz-base
\usepackage{listings}
\usepackage{color}    
\usepackage{array}    
\usepackage{longtable}
\usepackage{calc}     
\usepackage{multirow} 
\usepackage{hhline}   
\usepackage{ifthen}   
\usepackage{lscape}     
\usepackage{chngcntr}
\DeclareMathOperator*{\Res}{Res}
\renewcommand\thesection{\arabic{section}}
\renewcommand\thesubsection{\thesection.\arabic{subsection}}
\renewcommand\thesubsubsection{\thesubsection.\arabic{subsubsection}}

\renewcommand\thesectiondis{\arabic{section}}
\renewcommand\thesubsectiondis{\thesectiondis.\arabic{subsection}}
\renewcommand\thesubsubsectiondis{\thesubsectiondis.\arabic{subsubsection}}
\renewcommand\thetable{\arabic{table}}
% correct bad hyphenation here
\hyphenation{op-tical net-works semi-conduc-tor}
\def\inputGnumericTable{}                                 %%

\lstset{
%language=C,
frame=single, 
breaklines=true,
columns=fullflexible
}
%\lstset{
%language=tex,
%frame=single, 
%breaklines=true
%}

\begin{document}
\newtheorem{theorem}{Theorem}[section]
\newtheorem{problem}{Problem}
\newtheorem{proposition}{Proposition}[section]
\newtheorem{lemma}{Lemma}[section]
\newtheorem{corollary}[theorem]{Corollary}
\newtheorem{example}{Example}[section]
\newtheorem{definition}[problem]{Definition}
\newcommand{\BEQA}{\begin{eqnarray}}
\newcommand{\EEQA}{\end{eqnarray}}
\newcommand{\define}{\stackrel{\triangle}{=}}
\bibliographystyle{IEEEtran}
\providecommand{\mbf}{\mathbf}
\providecommand{\pr}[1]{\ensuremath{\Pr\left(#1\right)}}
\providecommand{\qfunc}[1]{\ensuremath{Q\left(#1\right)}}
\providecommand{\sbrak}[1]{\ensuremath{{}\left[#1\right]}}
\providecommand{\lsbrak}[1]{\ensuremath{{}\left[#1\right.}}
\providecommand{\rsbrak}[1]{\ensuremath{{}\left.#1\right]}}
\providecommand{\brak}[1]{\ensuremath{\left(#1\right)}}
\providecommand{\lbrak}[1]{\ensuremath{\left(#1\right.}}
\providecommand{\rbrak}[1]{\ensuremath{\left.#1\right)}}
\providecommand{\cbrak}[1]{\ensuremath{\left\{#1\right\}}}
\providecommand{\lcbrak}[1]{\ensuremath{\left\{#1\right.}}
\providecommand{\rcbrak}[1]{\ensuremath{\left.#1\right\}}}
\theoremstyle{remark}
\newtheorem{rem}{Remark}
\newcommand{\sgn}{\mathop{\mathrm{sgn}}}
\providecommand{\abs}[1]{\left\vert#1\right\vert}
\providecommand{\res}[1]{\Res\displaylimits_{#1}} 
\providecommand{\norm}[1]{\left\lVert#1\right\rVert}
\providecommand{\mtx}[1]{\mathbf{#1}}
\providecommand{\mean}[1]{E\left[ #1 \right]}
\providecommand{\fourier}{\overset{\mathcal{F}}{ \rightleftharpoons}}
\providecommand{\system}[1]{\overset{\mathcal{#1}}{ \longleftrightarrow}}
\newcommand{\solution}{\noindent \textbf{Solution: }}
\newcommand{\cosec}{\,\text{cosec}\,}
\providecommand{\dec}[2]{\ensuremath{\overset{#1}{\underset{#2}{\gtrless}}}}
\newcommand{\myvec}[1]{\ensuremath{\begin{pmatrix}#1\end{pmatrix}}}
\newcommand{\mydet}[1]{\ensuremath{\begin{vmatrix}#1\end{vmatrix}}}
\let\vec\mathbf
\def\putbox#1#2#3{\makebox[0in][l]{\makebox[#1][l]{}\raisebox{\baselineskip}[0in][0in]{\raisebox{#2}[0in][0in]{#3}}}}
     \def\rightbox#1{\makebox[0in][r]{#1}}
     \def\centbox#1{\makebox[0in]{#1}}
     \def\topbox#1{\raisebox{-\baselineskip}[0in][0in]{#1}}
     \def\midbox#1{\raisebox{-0.5\baselineskip}[0in][0in]{#1}}

\vspace{3cm}
\title{Probability Assignment}
\author{Gautam Singh}
\maketitle
\bigskip

\begin{abstract}
    This document contains the solution to Question 18 of 
    Exercise 2 in Chapter 13 of the class 12 NCERT textbook.
\end{abstract}

\begin{enumerate}
    \item Two events $A$ and $B$ will be independent, if
    \begin{enumerate}
        \item $A$ and $B$ are mutually exclusive
        \item $\pr{A'B'} = \brak{1 - \pr{A}}\brak{1 - \pr{B}}$
        \item $\pr{A} = \pr{B}$
        \item $\pr{A} + \pr{B} = 1$.
    \end{enumerate}

    \solution 
\fi
		Two events $A$ and $B$ are independent if
    \begin{align}
        \pr{AB} = \pr{A}\pr{B|A} = \pr{A}\pr{B} \label{eq:ncert/12/13/2/18/indep}
    \end{align}
    using Bayes' Rule.
    We consider the options one by one. Here, let $A$ be the event
    of rolling a prime number on a fair die and $B$ the event of
    rolling an odd prime number on a fair die. The joint pmf is shown
    in Table \ref{tab:ncert/12/13/2/18/joint}.
    \begin{table}[!ht]
        \centering
        \input{ncert/12/13/2/18/tables/12.13.02.18_1.tex}
        \caption{Joint PMF of $A$ and $B$.}
        \label{tab:ncert/12/13/2/18/joint}
    \end{table}
    Notice that $A$ and $B$ are independent, as
    \begin{align}
        \pr{A} = \frac{1}{2},\ \pr{B} = \frac{1}{3} \\
        \pr{AB} = \frac{1}{6} = \pr{A}\pr{B} \label{eq:ncert/12/13/2/18/nonzero}
    \end{align}
    thereby satisfying \eqref{eq:ncert/12/13/2/18/indep}
    \begin{enumerate}
        \item From \eqref{eq:ncert/12/13/2/18/nonzero}, $\pr{AB} > 0$, hence this option is
        incorrect.
        \item We have,
        \begin{align}
            &\pr{A'B'} = \pr{\brak{A+B}'} \label{eq:ncert/12/13/2/18/use-demorgan} \\
            &= 1 - \pr{A+B} \\
            &= 1 - \pr{A} - \pr{B} + \pr{AB} \\
            &= 1 - \pr{A} - \pr{B} + \pr{A}\pr{B} \label{eq:ncert/12/13/2/18/use-indep} \\
            &= \brak{1-\pr{A}}\brak{1-\pr{B}}
        \end{align}
        where \eqref{eq:ncert/12/13/2/18/use-demorgan} follows from De-Morgan's laws and
        \eqref{eq:ncert/12/13/2/18/use-indep} follows from \eqref{eq:ncert/12/13/2/18/indep}. Thus, this option is
        correct.
        \item Clearly from the given example, $\pr{A} \neq \pr{B}$. Thus, this
        option is incorrect.
        \item Again, from the given example, $\pr{A} + \pr{B} = \frac{5}{6} < 1$.
        Thus, this option is incorrect.
    \end{enumerate}
    Hence, the answer is option \textbf{b)}.
