\iffalse
\documentclass[12pt,twocolumn]{article}
\usepackage[margin=0.5in]{geometry}
\usepackage[cmex10]{amsmath}
\usepackage{amsmath}
\usepackage{amsmath,amssymb,amsfonts}
\usepackage{graphicx}
\usepackage{textcomp}
\usepackage{amsmath,amssymb,amsfonts,amsthm}
\usepackage{multirow}
\usepackage{adjustbox}
\usepackage{gensymb}
\newcommand*{\Comb}[2]{{}^{#1}C_{#2}}%
\let\vec\mathbf
\title{
Probability Assignment
}
\author{GINNA SHREYANI}
\date{}
\providecommand{\pr}[1]{\ensuremath{\Pr\left(#1\right)}}


\begin{document}
\maketitle

\textbf{12.13.3.3}\\
Of the students in a college, it is known that 60\% reside in hostel and 40\% are day scholars (not residing in hostel). Previous year results report that 30\% of all students who reside in hostel attain A grade and 20\% of day scholars attain A grade in their annual examination. At the end of the year, one student is chosen at random from the college and he has an A grade, what is the probability that the student is a hostlier?
\subsection*{Solution}
Using Baye's Rule:\\
Let the probability of students living in hostel be $\pr{H=1}$, therefore the students who are day scholars can be given as $\pr{H=0}$\\
The probability of the students getting grade A is given as $\pr{A=1}$.\\

\begin{table}
	%%%%%%%%%%%%%%%%%%%%%%%%%%%%%%%%%%%%%%%%%%%%%%%%%%%%%%%%%%%%%%%%%%%%%%
%%                                                                  %%
%%  This is a LaTeX2e table fragment exported from Gnumeric.        %%
%%                                                                  %%
%%%%%%%%%%%%%%%%%%%%%%%%%%%%%%%%%%%%%%%%%%%%%%%%%%%%%%%%%%%%%%%%%%%%%%

\begin{center}
\begin{tabular}{|c|c|c|}
\hline
\textbf{RV}& \textbf{Values} & \textbf{Description} \\ \hline
$X$		   & 	$\{0,1\}$	&  1st draw - 0: Red, 1: Black\\ \hline
$Y$ 		   & 	$\{0,1\}$	&  2nd draw - 0: Red, 1: Black\\ \hline
\end{tabular}
\end{center}

\end{table}

\begin{table}
        %%%%%%%%%%%%%%%%%%%%%%%%%%%%%%%%%%%%%%%%%%%%%%%%%%%%%%%%%%%%%%%%%%%%%%
%%                                                                  %%
%%  This is a LaTeX2e table fragment exported from Gnumeric.        %%
%%                                                                  %%
%%%%%%%%%%%%%%%%%%%%%%%%%%%%%%%%%%%%%%%%%%%%%%%%%%%%%%%%%%%%%%%%%%%%%%

\begin{center}
\begin{tabular}{|c|c|}
\hline
\textbf{Event}& \textbf{Probability} \\ \hline
$\pr{A = 1}$ & 	$\frac{4}{5}$ \\ \hline
$\pr{X = 1}$ & 	$\frac{1}{2}$ \\ \hline
$\pr{X = 1 \mid A= 1}$ &  $\frac{1}{2}$ \\ \hline
\end{tabular}
\end{center}

\end{table}

Thus,
\begin{align}
	\pr{A=1} &= \sum_{i=0}^1 \pr{A = 1 \mid H = i}\pr{H=i}
\end{align}
\begin{multline}
	\pr{A=1} = \pr{A = 1 \mid H = 0}\pr{H = 0}\\
	+\pr{A = 1 \mid H = 1}\pr{H = 1}
\end{multline}
\begin{align}
	\pr{A=1} &= \left(\frac{20}{100} \times \frac{40}{100}\right)+\left(\frac{30}{100} \times \frac{60}{100}\right)\\
	\pr{A=1} &= \frac{26}{100}\\
	\pr{H=1 \mid A=1} &= \frac{\pr{A=1 \mid H=1}\pr{H=1}}{\pr{A=1}}\\
	\pr{H=1 \mid A=1} &= \frac{9}{13}
\end{align}
The probability that the student is a hostlier who has A grade is $\frac{9}{13}$.
\end{document}

