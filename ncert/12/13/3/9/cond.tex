\iffalse
%\documentclass[class=article, crop=false]{standalone}
\documentclass{article}
\usepackage{amssymb,amsfonts,amsthm,amsmath}
\usepackage{enumitem}
\usepackage{hyperref,xcolor}
\hypersetup{
    colorlinks,
    urlcolor={black}	%black!50!blue
}
\providecommand{\pr}[1]{\ensuremath{\Pr\left(#1\right)}}
%\documentclass{article}	% working
\def\inputGnumericTable{}
\usepackage[latin1]{inputenc}
\usepackage{fullpage}
\usepackage{color}
\usepackage{array}
\usepackage{longtable}
\usepackage{calc}
\usepackage{multirow}
\usepackage{hhline}
\usepackage{ifthen}
\providecommand{\cbrak}[1]{\ensuremath{\left\{#1\right\}}}
\newcommand{\solution}{\noindent \textbf{Solution: }}
\providecommand{\pr}[1]{\ensuremath{\Pr\left(#1\right)}}
%\newcommand{\varsol}{\noindent \textbf{Aliter: }}
\newcommand*{\permcomb}[4][0mu]{{{}^{#3}\mkern#1#2_{#4}}}
%\newcommand*{\perm}[1][-3mu]{\permcomb[#1]{P}}
\newcommand*{\comb}[1][-1mu]{\permcomb[#1]{C}}
\setlist[enumerate]{font=\small\bfseries}
\renewcommand\thefootnote{\textcolor{black}{\arabic{footnote}}}

\begin{document}

\title{PROBABILITY}
\author{\Large HARI VENKATESWARLU - FWC22058}
\date{}

\maketitle

\begin{enumerate}[label=13.\arabic{enumi}.\arabic{enumii}]%,ref=\thesection.\theenumi.\theenumi]
\numberwithin{equation}{enumi}
\setcounter{enumi}{0}
\setcounter{enumii}{11}

\item \footnote{Read question numbers as (CHAPTER NUMBER).(EXERCISE NUMBER).(QUESTION NUMBER)} {Two groups are competing for the position
on the Board of directors of a corporation.
The probabilities that the first and the second
groups will win are 0.6 and 0.4 respectively.
Further, if the first group wins, the probability
of introducing a new product is 0.7 and the
corresponding probability is 0.3 if the second
group wins. Find the probability that the new
product introduced was by the second group}

	\solution\\
	\fi
	The given information is listed in Tables 
	\ref{tab:ncert/12/13/3/9/1}
	and 
	\ref{tab:ncert/12/13/3/9/2}
	\begin{table}[h]\centering
	
%%%%%%%%%%%%%%%%%%%%%%%%%%%%%%%%%%%%%%%%%%%%%%%%%%%%%%%%%%%%%%%%%%%%%%
%%                                                                  %%
%%  This is a LaTeX2e table fragment exported from Gnumeric.        %%
%%                                                                  %%
%%%%%%%%%%%%%%%%%%%%%%%%%%%%%%%%%%%%%%%%%%%%%%%%%%%%%%%%%%%%%%%%%%%%%%

\begin{center}
\begin{tabular}{|c|c|l|}
\hline
 Parameter&  Value &Description\\ \hline
 $X$ & $bin(n,p)$ & no of correct answers  that candidate gets by guessing\\\hline
$n$ &  5 & total no of questions\\ \hline
 $p$ &  $\frac{1}{3}$ & probability of getting correct answer by guessing\\\hline

\end{tabular}
\end{center}
	\caption{Random variables(RV) X,Y}
	\label{tab:ncert/12/13/3/9/1}
\end{table}

    \begin{table}[h]\centering
	%%%%%%%%%%%%%%%%%%%%%%%%%%%%%%%%%%%%%%%%%%%%%%%%%%%%%%%%%%%%%%%%%%%%%%
%%                                                                  %%
%%  This is a LaTeX2e table fragment exported from Gnumeric.        %%
%%                                                                  %%
%%%%%%%%%%%%%%%%%%%%%%%%%%%%%%%%%%%%%%%%%%%%%%%%%%%%%%%%%%%%%%%%%%%%%%

\begin{center}
\begin{tabular}{|c|c|c|}
\hline
\textbf{RV}& \textbf{Values} & \textbf{Description} \\ \hline
$X$		   & 	$\{0,1\}$	&  1st draw - 0: Red, 1: Black\\ \hline
$Y$ 		   & 	$\{0,1\}$	&  2nd draw - 0: Red, 1: Black\\ \hline
\end{tabular}
\end{center}

	\caption{Probabilities}
	\label{tab:ncert/12/13/3/9/2}
    \end{table}
\begin{align}
\pr{X=2 \mid Y=1} = \frac{\pr{2}\pr{1 \mid 2}}{\pr{1}\pr{1 \mid 1}+\pr{2}\pr{1 \mid 2}}
=\frac{2}{9}
\end{align}
