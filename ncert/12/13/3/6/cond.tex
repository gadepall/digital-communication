\iffalse
\documentclass[journal,12pt,twocolumn]{IEEEtran}
\usepackage{setspace}
\usepackage{gensymb}
\singlespacing
\usepackage[cmex10]{amsmath}
\usepackage{amsthm}
\usepackage{mathrsfs}
\usepackage{txfonts}
\usepackage{stfloats}
\usepackage{bm}
\usepackage{cite}
\usepackage{cases}
\usepackage{subfig}
\usepackage{longtable}
\usepackage{multirow}
\usepackage{enumitem}
\usepackage{mathtools}
\usepackage{tikz}
\usepackage{circuitikz}
\usepackage{verbatim}
\usepackage[breaklinks=true]{hyperref}
\usepackage{tkz-euclide} % loads  TikZ and tkz-base
\usepackage{listings}
\usepackage{color}    
\usepackage{array}    
\usepackage{longtable}
\usepackage{calc}     
\usepackage{multirow} 
\usepackage{hhline}   
\usepackage{ifthen}   
\usepackage{lscape}     
\usepackage{chngcntr}
\DeclareMathOperator*{\Res}{Res}
\renewcommand\thesection{\arabic{section}}
\renewcommand\thesubsection{\thesection.\arabic{subsection}}
\renewcommand\thesubsubsection{\thesubsection.\arabic{subsubsection}}

\renewcommand\thesectiondis{\arabic{section}}
\renewcommand\thesubsectiondis{\thesectiondis.\arabic{subsection}}
\renewcommand\thesubsubsectiondis{\thesubsectiondis.\arabic{subsubsection}}
\renewcommand\thetable{\arabic{table}}
% correct bad hyphenation here
\hyphenation{op-tical net-works semi-conduc-tor}
\def\inputGnumericTable{}                                 %%

\lstset{
%language=C,
frame=single, 
breaklines=true,
columns=fullflexible
}
%\lstset{
%language=tex,
%frame=single, 
%breaklines=true
%}

\begin{document}
\newtheorem{theorem}{Theorem}[section]
\newtheorem{problem}{Problem}
\newtheorem{proposition}{Proposition}[section]
\newtheorem{lemma}{Lemma}[section]
\newtheorem{corollary}[theorem]{Corollary}
\newtheorem{example}{Example}[section]
\newtheorem{definition}[problem]{Definition}
\newcommand{\BEQA}{\begin{eqnarray}}
\newcommand{\EEQA}{\end{eqnarray}}
\newcommand{\define}{\stackrel{\triangle}{=}}
\bibliographystyle{IEEEtran}
\providecommand{\mbf}{\mathbf}
\providecommand{\pr}[1]{\ensuremath{\Pr\left(#1\right)}}
\providecommand{\qfunc}[1]{\ensuremath{Q\left(#1\right)}}
\providecommand{\sbrak}[1]{\ensuremath{{}\left[#1\right]}}
\providecommand{\lsbrak}[1]{\ensuremath{{}\left[#1\right.}}
\providecommand{\rsbrak}[1]{\ensuremath{{}\left.#1\right]}}
\providecommand{\brak}[1]{\ensuremath{\left(#1\right)}}
\providecommand{\lbrak}[1]{\ensuremath{\left(#1\right.}}
\providecommand{\rbrak}[1]{\ensuremath{\left.#1\right)}}
\providecommand{\cbrak}[1]{\ensuremath{\left\{#1\right\}}}
\providecommand{\lcbrak}[1]{\ensuremath{\left\{#1\right.}}
\providecommand{\rcbrak}[1]{\ensuremath{\left.#1\right\}}}
\theoremstyle{remark}
\newtheorem{rem}{Remark}
\newcommand{\sgn}{\mathop{\mathrm{sgn}}}
\providecommand{\abs}[1]{\left\vert#1\right\vert}
\providecommand{\res}[1]{\Res\displaylimits_{#1}} 
\providecommand{\norm}[1]{\left\lVert#1\right\rVert}
\providecommand{\mtx}[1]{\mathbf{#1}}
\providecommand{\mean}[1]{E\left[ #1 \right]}
\providecommand{\fourier}{\overset{\mathcal{F}}{ \rightleftharpoons}}
\providecommand{\system}[1]{\overset{\mathcal{#1}}{ \longleftrightarrow}}
\newcommand{\solution}{\noindent \textbf{Solution: }}
\newcommand{\cosec}{\,\text{cosec}\,}
\providecommand{\dec}[2]{\ensuremath{\overset{#1}{\underset{#2}{\gtrless}}}}
\newcommand{\myvec}[1]{\ensuremath{\begin{pmatrix}#1\end{pmatrix}}}
\newcommand{\mydet}[1]{\ensuremath{\begin{vmatrix}#1\end{vmatrix}}}
\let\vec\mathbf
\def\putbox#1#2#3{\makebox[0in][l]{\makebox[#1][l]{}\raisebox{\baselineskip}[0in][0in]{\raisebox{#2}[0in][0in]{#3}}}}
     \def\rightbox#1{\makebox[0in][r]{#1}}
     \def\centbox#1{\makebox[0in]{#1}}
     \def\topbox#1{\raisebox{-\baselineskip}[0in][0in]{#1}}
     \def\midbox#1{\raisebox{-0.5\baselineskip}[0in][0in]{#1}}

\vspace{3cm}
\title{Probability Assignment}
\author{Gautam Singh}
\maketitle
\bigskip

\begin{abstract}
    This document contains the solution to Question 6 of 
    Exercise 3 in Chapter 13 of the class 12 NCERT textbook.
\end{abstract}

\begin{enumerate}
    \item There are three coins. One is a two headed coin (having head on both 
    faces), another is a biased coin that comes up heads 75\% of the time and 
    third is an unbiased coin. One of the three coins is chosen at random and 
    tossed, it shows heads, what is the probability that it was the two headed 
    coin?

    \solution 
\fi
		Define the random variable $X$ as in Table \ref{tab:ncert/12/13/3/6/X-def}.
    \begin{table}[!ht]
        \centering
        \input{ncert/12/13/3/6/tables/12.13.03.06_1.tex}
        \caption{Definition of $X$.}
        \label{tab:ncert/12/13/3/6/X-def}
    \end{table}
    Clearly, the pmf of $X$ is
    \begin{align}
        \pr{X=k} =
        \begin{cases}
            \frac{1}{3} & 1 \le k \le 3 \\
            0 & \textrm{otherwise}
        \end{cases} 
        \label{eq:ncert/12/13/3/6/X-pmf}
    \end{align}
    Let the random variables $Y_1$, $Y_2$ and $Y_3$ (one for each coin) be
    defined as
    \begin{align}
        Y_1 &\sim \textrm{Ber}\brak{1} \label{eq:ncert/12/13/3/6/Y-1-def} \\
        Y_2 &\sim \textrm{Ber}\brak{\frac{3}{4}} \label{eq:ncert/12/13/3/6/Y-2-def} \\
        Y_3 &\sim \textrm{Ber}\brak{\frac{1}{2}} \label{eq:ncert/12/13/3/6/Y-3-def}
    \end{align}
    Define $Y$ as
    \begin{align}
        Y \triangleq \sum_{i=1}^3\vec{1}_i\brak{X}Y_i
        \label{eq:ncert/12/13/3/6/Y-def}
    \end{align}
    where $\vec{1}$ denotes the indicator random variable, defined as
    \begin{align}
        \vec{1}_i\brak{X} = 
        \begin{cases}
            1 & \textrm{if $X=i$} \\
            0 & \textrm{otherwise}
        \end{cases}
    \end{align}
    We are required to find $\pr{X=1|Y=1}$. However, from Bayes' Rule,
    \begin{align}
        \pr{X=1,Y=1} &= \pr{X=1}\pr{Y=1|X=1} \label{eq:ncert/12/13/3/6/e1} \\
                       &= \pr{Y=1}\pr{X=1|Y=1} \label{eq:ncert/12/13/3/6/e2}
    \end{align}
    Note from \eqref{eq:ncert/12/13/3/6/Y-def} that
    \begin{align}
        X=1 \implies Y=Y_1
    \end{align}
    and also,
    \begin{align}
        \pr{Y=1} &= \sum_{i=1}^3\pr{X=i}\pr{Y_i=1} \\
                 &= \frac{1}{3}\brak{1+\frac{3}{4}+\frac{1}{2}} = \frac{3}{4}
                 \label{eq:ncert/12/13/3/6/pr-Y1}
    \end{align}
    Thus, from \eqref{eq:ncert/12/13/3/6/e1}, \eqref{eq:ncert/12/13/3/6/e2} and \eqref{eq:ncert/12/13/3/6/pr-Y1}, we see that
    \begin{align}
        \pr{X=1|Y=1} &= \frac{\pr{X=1}\pr{Y_1=1}}{\pr{Y=1}} \\
                     &= \frac{\frac{1}{3}}{\frac{3}{4}} = \frac{4}{9}
    \end{align}
