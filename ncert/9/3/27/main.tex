\iffalse
\let\negmedspace\undefined
\let\negthickspace\undefined
\documentclass[journal,12pt,twocolumn]{IEEEtran}
\usepackage{cite}
\usepackage{amsmath,amssymb,amsfonts,amsthm}
\usepackage{algorithmic}
\usepackage{graphicx}
\usepackage{textcomp}
\usepackage{xcolor}
\usepackage{txfonts}
\usepackage{listings}
\usepackage{enumitem}
\usepackage{mathtools}
\usepackage{gensymb}
\usepackage[breaklinks=true]{hyperref}
\usepackage{tkz-euclide} % loads  TikZ and tkz-base
\usepackage{listings}


\newtheorem{theorem}{Theorem}[section]
\newtheorem{problem}{Problem}
\newtheorem{proposition}{Proposition}[section]
\newtheorem{lemma}{Lemma}[section]
\newtheorem{corollary}[theorem]{Corollary}
\newtheorem{example}{Example}[section]
\newtheorem{definition}[problem]{Definition}
%\newtheorem{thm}{Theorem}[section] 
%\newtheorem{defn}[thm]{Definition}
%\newtheorem{algorithm}{Algorithm}[section]
%\newtheorem{cor}{Corollary}
\newcommand{\BEQA}{\begin{eqnarray}}
\newcommand{\EEQA}{\end{eqnarray}}
\newcommand{\define}{\stackrel{\triangle}{=}}
\theoremstyle{remark}
\newtheorem{rem}{Remark}

%\bibliographystyle{ieeetr}
\begin{document}
%

\providecommand{\pr}[1]{\ensuremath{\Pr\left(#1\right)}}
\providecommand{\prt}[2]{\ensuremath{p_{#1}^{\left(#2\right)} }}        % own macro for this question
\providecommand{\qfunc}[1]{\ensuremath{Q\left(#1\right)}}
\providecommand{\sbrak}[1]{\ensuremath{{}\left[#1\right]}}
\providecommand{\lsbrak}[1]{\ensuremath{{}\left[#1\right.}}
\providecommand{\rsbrak}[1]{\ensuremath{{}\left.#1\right]}}
\providecommand{\brak}[1]{\ensuremath{\left(#1\right)}}
\providecommand{\lbrak}[1]{\ensuremath{\left(#1\right.}}
\providecommand{\rbrak}[1]{\ensuremath{\left.#1\right)}}
\providecommand{\cbrak}[1]{\ensuremath{\left\{#1\right\}}}
\providecommand{\lcbrak}[1]{\ensuremath{\left\{#1\right.}}
\providecommand{\rcbrak}[1]{\ensuremath{\left.#1\right\}}}
\newcommand{\sgn}{\mathop{\mathrm{sgn}}}
\providecommand{\abs}[1]{\left\vert#1\right\vert}
\providecommand{\res}[1]{\Res\displaylimits_{#1}} 
\providecommand{\norm}[1]{\left\lVert#1\right\rVert}
%\providecommand{\norm}[1]{\lVert#1\rVert}
\providecommand{\mtx}[1]{\mathbf{#1}}
\providecommand{\mean}[1]{E\left[ #1 \right]}
\providecommand{\cond}[2]{#1\middle|#2}
\providecommand{\fourier}{\overset{\mathcal{F}}{ \rightleftharpoons}}
\newenvironment{amatrix}[1]{%
  \left(\begin{array}{@{}*{#1}{c}|c@{}}
}{%
  \end{array}\right)
}
%\providecommand{\hilbert}{\overset{\mathcal{H}}{ \rightleftharpoons}}
%\providecommand{\system}{\overset{\mathcal{H}}{ \longleftrightarrow}}
	%\newcommand{\solution}[2]{\textbf{Solution:}{#1}}
\newcommand{\solution}{\noindent \textbf{Solution: }}
\newcommand{\cosec}{\,\text{cosec}\,}
\providecommand{\dec}[2]{\ensuremath{\overset{#1}{\underset{#2}{\gtrless}}}}
\newcommand{\myvec}[1]{\ensuremath{\begin{pmatrix}#1\end{pmatrix}}}
\newcommand{\mydet}[1]{\ensuremath{\begin{vmatrix}#1\end{vmatrix}}}
\newcommand{\myaugvec}[2]{\ensuremath{\begin{amatrix}{#1}#2\end{amatrix}}}
\providecommand{\rank}{\text{rank}}
\providecommand{\pr}[1]{\ensuremath{\Pr\left(#1\right)}}
\providecommand{\qfunc}[1]{\ensuremath{Q\left(#1\right)}}
	\newcommand*{\permcomb}[4][0mu]{{{}^{#3}\mkern#1#2_{#4}}}
\newcommand*{\perm}[1][-3mu]{\permcomb[#1]{P}}
\newcommand*{\comb}[1][-1mu]{\permcomb[#1]{C}}
\providecommand{\qfunc}[1]{\ensuremath{Q\left(#1\right)}}
\providecommand{\gauss}[2]{\mathcal{N}\ensuremath{\left(#1,#2\right)}}
\providecommand{\diff}[2]{\ensuremath{\frac{d{#1}}{d{#2}}}}
\providecommand{\myceil}[1]{\left \lceil #1 \right \rceil }
\newcommand\figref{Fig.~\ref}
\newcommand\tabref{Table~\ref}
\newcommand{\sinc}{\,\text{sinc}\,}
\newcommand{\rect}{\,\text{rect}\,}
%%
%	%\newcommand{\solution}[2]{\textbf{Solution:}{#1}}
%\newcommand{\solution}{\noindent \textbf{Solution: }}
%\newcommand{\cosec}{\,\text{cosec}\,}
%\numberwithin{equation}{section}
%\numberwithin{equation}{subsection}
%\numberwithin{problem}{section}
%\numberwithin{definition}{section}
%\makeatletter
%\@addtoreset{figure}{problem}
%\makeatother

%\let\StandardTheFigure\thefigure
\let\vec\mathbf

\bibliographystyle{IEEEtran}


\vspace{3cm}

\title{

\textbf{QUESTION : 12.13.3.7} 

}
\author{ ROLL NO:EE22BTECH11027\\
         NAME: KATARI SIRI VARSHINI
	}
	
	

\maketitle

\newpage

%\tableofcontents

\bigskip

\renewcommand{\thefigure}{\theenumi}
\renewcommand{\thetable}{\theenumi}
9.3.27.The probability of a man hitting the target is 0.25. He shoots 7 times.What is the probability of his hitting the target atleast twice?
\fi
\\ \solution:
\begin{table}[ht]
    \centering
    \caption{Variables}
    \label{table:gaussian/9/3/27/}
\begin{tabular}{|c|c|c|}
\hline
Variable & Value & Description \\
\hline
n & 7 & Number of trails\\
\hline
p & 0.25 & The probability of man hitting the target\\
\hline
q & 0.75 & The probability of man not hitting the target\\
\hline
$\mu=np$ & 1.75 & mean of distribution\\
\hline
$\sigma=\sqrt{npq}$ & 1.145 & variance of distribution\\
\hline
$X$ & $ X \ge 2 $ & Number of times man hits the target \\
\hline
\end{tabular} 
\end{table}\\
From gaussian,
\begin{align}
Y \sim \gauss{\mu}{\sigma^{2}}
\end{align}
CDF of Y is defined as:
\begin{align}
	F_Y\brak{X}&=\pr{Y<X}\\
	&=\pr{\frac{Y-\mu}{\sigma}\le \frac{X-\mu}{\sigma}}\\
	\implies \frac{Y-\mu}{\sigma} &\sim N\brak{0,1}\\
	&=1-\pr{\frac{Y-\mu}{\sigma}>\frac{X-\mu}{\sigma}}\\
	&=\begin{cases}
		1-\qfunc{\frac{X-\mu}{\sigma}} & X \geq \mu\\
		\qfunc{\frac{\mu -X}{\sigma}} & X < \mu 
	\end{cases}
\end{align}
Hence, the probability of hitting target atleast twice using gaussian distribution is:\\
without correction:
\begin{align}
\pr{Y \ge 2} &= 1-\pr{Y < 2}\\
&= 1-F_Y\brak{2}\\
\implies \pr{Y \ge 2} &= \qfunc{\frac{X-\mu}{\sigma}}\\
&= \qfunc{0.218}\\
\pr{Y \ge 2} &= 0.4137
\end{align}
with correction:
\begin{align}
\pr{Y \ge 2} &= \qfunc{\frac{X-0.5-\mu}{\sigma}}\\
&= \qfunc{-0.218}\\
\pr{Y \ge 2} &= 0.5862
\end{align}
Hence, the probability of hitting target atleast twice using binomial distribution is:
\begin{align}
\pr{X\ge 2} &= 1 - \pr{X<2} \\
&= 1-\sum_{k=0}^{1}\comb{n}{k}p^k\brak{1-p}^{n-k} \\
&= 0.55
\end{align}
\begin{figure}[h]
\includegraphics[width=\columnwidth]{ncert/9/3/27/figure/9.3.27.png}
\caption{gaussian and binomial}
\label{fig1:gaussian/9/3/27/}
\end{figure}


