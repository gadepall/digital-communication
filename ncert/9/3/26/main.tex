\iffalse
\documentclass[journal,12pt,onecolumn]{IEEEtran}
\usepackage{setspace}
\usepackage{gensymb}
\singlespacing
\usepackage[cmex10]{amsmath}
\usepackage{amsthm}
\usepackage{mathrsfs}
\usepackage{txfonts}
\usepackage{stfloats}
\usepackage{bm}
\usepackage{cite}
\usepackage{cases}
\usepackage{subfig}
\usepackage{longtable}
\usepackage{multirow}
\usepackage{enumitem}
\usepackage{mathtools}
\usepackage{tikz}
\usepackage{circuitikz}
\usepackage{verbatim}
\usepackage[breaklinks=true]{hyperref}
\usepackage{tkz-euclide} % loads  TikZ and tkz-base
\usepackage{listings}
\usepackage{color}    
\usepackage{array}    
\usepackage{longtable}
\usepackage{calc}     

\usepackage{hhline}   
\usepackage{ifthen}   
\usepackage{lscape}     
\usepackage{chngcntr}
\usepackage{float}
\DeclareMathOperator*{\Res}{Res}
\renewcommand\thesection{\arabic{section}}
\renewcommand\thesubsection{\thesection.\arabic{subsection}}
\renewcommand\thesubsubsection{\thesubsection.\arabic{subsubsection}}

\renewcommand\thesectiondis{\arabic{section}}
\renewcommand\thesubsectiondis{\thesectiondis.\arabic{subsection}}
\renewcommand\thesubsubsectiondis{\thesubsectiondis.\arabic{subsubsection}}
\renewcommand\thetable{\arabic{table}}
% correct bad hyphenation here
\hyphenation{op-tical net-works semi-conduc-tor}
\def\inputGnumericTable{}                                 %%

\lstset{
%language=C,
frame=single, 
breaklines=true,
columns=fullflexible
}
%\lstset{
%language=tex,
%frame=single, 
%breaklines=true
%}

\begin{document}
\newtheorem{theorem}{Theorem}[section]
\newtheorem{problem}{Problem}
\newtheorem{proposition}{Proposition}[section]
\newtheorem{lemma}{Lemma}[section]
\newtheorem{corollary}[theorem]{Corollary}
\newtheorem{example}{Example}[section]
\newtheorem{definition}[problem]{Definition}
\newcommand{\BEQA}{\begin{eqnarray}}
\newcommand{\EEQA}{\end{eqnarray}}
\newcommand{\define}{\stackrel{\triangle}{=}}
\theoremstyle{remark}
\newtheorem{rem}{Remark}
\bibliographystyle{IEEEtran}
\providecommand{\mbf}{\mathbf}

\providecommand{\pr}[1]{\ensuremath{\Pr\left(#1\right)}}
\providecommand{\prt}[2]{\ensuremath{p_{#1}^{\left(#2\right)} }}        % own macro for this question
\providecommand{\qfunc}[1]{\ensuremath{Q\left(#1\right)}}
\providecommand{\sbrak}[1]{\ensuremath{{}\left[#1\right]}}
\providecommand{\lsbrak}[1]{\ensuremath{{}\left[#1\right.}}
\providecommand{\rsbrak}[1]{\ensuremath{{}\left.#1\right]}}
\providecommand{\brak}[1]{\ensuremath{\left(#1\right)}}
\providecommand{\lbrak}[1]{\ensuremath{\left(#1\right.}}
\providecommand{\rbrak}[1]{\ensuremath{\left.#1\right)}}
\providecommand{\cbrak}[1]{\ensuremath{\left\{#1\right\}}}
\providecommand{\lcbrak}[1]{\ensuremath{\left\{#1\right.}}
\providecommand{\rcbrak}[1]{\ensuremath{\left.#1\right\}}}
\newcommand{\sgn}{\mathop{\mathrm{sgn}}}
\providecommand{\abs}[1]{\left\vert#1\right\vert}
\providecommand{\res}[1]{\Res\displaylimits_{#1}} 
\providecommand{\norm}[1]{\left\lVert#1\right\rVert}
%\providecommand{\norm}[1]{\lVert#1\rVert}
\providecommand{\mtx}[1]{\mathbf{#1}}
\providecommand{\mean}[1]{E\left[ #1 \right]}
\providecommand{\cond}[2]{#1\middle|#2}
\providecommand{\fourier}{\overset{\mathcal{F}}{ \rightleftharpoons}}
\newenvironment{amatrix}[1]{%
  \left(\begin{array}{@{}*{#1}{c}|c@{}}
}{%
  \end{array}\right)
}

\newcommand{\solution}{\noindent \textbf{Solution: }}
\newcommand{\cosec}{\,\text{cosec}\,}
\providecommand{\dec}[2]{\ensuremath{\overset{#1}{\underset{#2}{\gtrless}}}}
\newcommand{\myvec}[1]{\ensuremath{\begin{pmatrix}#1\end{pmatrix}}}
\newcommand{\mydet}[1]{\ensuremath{\begin{vmatrix}#1\end{vmatrix}}}
\newcommand{\myaugvec}[2]{\ensuremath{\begin{amatrix}{#1}#2\end{amatrix}}}
\providecommand{\rank}{\text{rank}}
\providecommand{\pr}[1]{\ensuremath{\Pr\left(#1\right)}}
\providecommand{\qfunc}[1]{\ensuremath{Q\left(#1\right)}}
	\newcommand*{\permcomb}[4][0mu]{{{}^{#3}\mkern#1#2_{#4}}}
\newcommand*{\perm}[1][-3mu]{\permcomb[#1]{P}}
\newcommand*{\comb}[1][-1mu]{\permcomb[#1]{C}}
\providecommand{\qfunc}[1]{\ensuremath{Q\left(#1\right)}}
\providecommand{\gauss}[2]{\mathcal{N}\ensuremath{\left(#1,#2\right)}}
\providecommand{\diff}[2]{\ensuremath{\frac{d{#1}}{d{#2}}}}
\providecommand{\myceil}[1]{\left \lceil #1 \right \rceil }
\newcommand\figref{Fig.~\ref}
\newcommand\tabref{Table~\ref}
\newcommand{\sinc}{\,\text{sinc}\,}
\newcommand{\rect}{\,\text{rect}\,}
\let\vec\mathbf
\vspace{3cm}
\title{}
\author{EE22BTECH11032 - Meenakshi}
\maketitle
\textbf{Question 9.3.26}
Question: A lot of 100 watches is known to have 10 defective watches. If 8 watches are selected
(one by one with replacement) at random, what is the probability that there will be
at least one defective watch?
\solution
\fi
\begin{table}[!htb]
	%%%%%%%%%%%%%%%%%%%%%%%%%%%%%%%%%%%%%%%%%%%%%%%%%%%%%%%%%%%%%%%%%%%%%%
%%                                                                  %%
%%  This is a LaTeX2e table fragment exported from Gnumeric.        %%
%%                                                                  %%
%%%%%%%%%%%%%%%%%%%%%%%%%%%%%%%%%%%%%%%%%%%%%%%%%%%%%%%%%%%%%%%%%%%%%%

\begin{center}
\begin{tabular}{|c|c|c|}
\hline
\textbf{RV}& \textbf{Values} & \textbf{Description} \\ \hline
$X$		   & 	$\{0,1\}$	&  1st draw - 0: Red, 1: Black\\ \hline
$Y$ 		   & 	$\{0,1\}$	&  2nd draw - 0: Red, 1: Black\\ \hline
\end{tabular}
\end{center}

	\caption{Gaussian Info Table}
	\label{table:gaussian/9/3/26/}	
\end{table}
\begin{enumerate}[label=(\roman*)]
\item Gaussian Distribution\\
Let Y is the Gaussian obtained by approximating binomial with parameters n,p then By Central limit theroem, 
\begin{align}
	Y \sim \mathcal{N}(np,npq)
\end{align}
CDF of $Y$ is:
\begin{align}
	F_Y(x)&=\pr{Y \le x} \\
	&=\pr{Y-\mu \le x-\mu} \\
	&=\pr{\frac{Y-\mu}{\sigma} \le \frac{x-\mu}{\sigma}} 
\end{align}
Since, 
\begin{align}
	\frac{Y-\mu}{\sigma} \sim \mathcal{N}(0,1)
\end{align}
Q function is defined
\begin{align}
	\qfunc{x} &= \pr{Y > x} \, \forall x \in Y \sim \mathcal{N}(0,1) 
\end{align}
\begin{align}
	F_Y(x)&=1-\pr{\frac{Y-\mu}{\sigma} > \frac{x-\mu}{\sigma}}\\
	&= 
    \begin{cases}
        1-\qfunc{ \frac{x-\mu}{\sigma}}, &  x > \mu \\
        \qfunc{ \frac{\mu-x}{\sigma}} , &  x < \mu
    \end{cases} 
\end{align}
\begin{enumerate}[label=(\alph*)]
\item For atleast one watch to be defective, we need to find
\begin{align}
        1 - \pr{Y=0}
\end{align}
\begin{align}
	\pr{Y = 0} &= \pr{Y \le 1} \\
		   &= F_Y(1) 
\end{align}
\begin{align}
	F_Y(1)&=1 -\qfunc{ \frac{1-0.8}{0.848}} \\
	&= 1 - \qfunc{0.235}\\
	&= 0.58\\
	\pr{Y = 0} &= F_Y(1)\\
		   &= 0.58
\end{align}
\end{enumerate}
\item Binomial Distribution\\
Lets define a random variable $X$ which represents the number of defective bulbs.
\begin{align}
X = \{0,1,2,3,4,5,6,7,8,9,10\}
\end{align}
The pmf is given by 
\begin{align} 
P_X\brak{r} &= \comb{n}{r} p^{r}(1-p)^{n-r}
\end{align}
If we consider atleast one watch to be defective, we need, 
\begin{align}
1 - P_X\brak{0}\\
P_X\brak{0} = 0.430\\
1 - P_X\brak{0} = 0.569
\end{align}
\item Binomial vs Gaussian Graph\\
\includegraphics[width=\columnwidth]{ncert/9/3/26/figs/plot.pdf}
\end{enumerate}
