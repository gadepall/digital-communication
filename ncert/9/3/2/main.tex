\iffalse
\documentclass[journal,12pt,onecolumn]{IEEEtran}
\usepackage{setspace}
\usepackage{gensymb}
\singlespacing
\usepackage[cmex10]{amsmath}
\usepackage{amsthm}
\usepackage{mathrsfs}
\usepackage{txfonts}
\usepackage{stfloats}
\usepackage{bm}
\usepackage{cite}
\usepackage{cases}
\usepackage{subfig}
\usepackage{longtable}
\usepackage{multirow}
\usepackage{enumitem}
\usepackage{mathtools}
\usepackage{tikz}
\usepackage{circuitikz}
\usepackage{verbatim}
\usepackage[breaklinks=true]{hyperref}
\usepackage{tkz-euclide} % loads  TikZ and tkz-base
\usepackage{listings}
\usepackage{color}    
\usepackage{array}    
\usepackage{longtable}
\usepackage{calc}     

\usepackage{hhline}   
\usepackage{ifthen}   
\usepackage{lscape}     
\usepackage{chngcntr}
\usepackage{float}
\DeclareMathOperator*{\Res}{Res}
\renewcommand\thesection{\arabic{section}}
\renewcommand\thesubsection{\thesection.\arabic{subsection}}
\renewcommand\thesubsubsection{\thesubsection.\arabic{subsubsection}}

\renewcommand\thesectiondis{\arabic{section}}
\renewcommand\thesubsectiondis{\thesectiondis.\arabic{subsection}}
\renewcommand\thesubsubsectiondis{\thesubsectiondis.\arabic{subsubsection}}
\renewcommand\thetable{\arabic{table}}
% correct bad hyphenation here
\hyphenation{op-tical net-works semi-conduc-tor}
\def\inputGnumericTable{}                                 %%

\lstset{
%language=C,
frame=single, 
breaklines=true,
columns=fullflexible
}
%\lstset{
%language=tex,
%frame=single, 
%breaklines=true
%}

\begin{document}
\newtheorem{theorem}{Theorem}[section]
\newtheorem{problem}{Problem}
\newtheorem{proposition}{Proposition}[section]
\newtheorem{lemma}{Lemma}[section]
\newtheorem{corollary}[theorem]{Corollary}
\newtheorem{example}{Example}[section]
\newtheorem{definition}[problem]{Definition}
\newcommand{\BEQA}{\begin{eqnarray}}
\newcommand{\EEQA}{\end{eqnarray}}
\newcommand{\define}{\stackrel{\triangle}{=}}
\theoremstyle{remark}
\newtheorem{rem}{Remark}
\bibliographystyle{IEEEtran}
\providecommand{\mbf}{\mathbf}

\providecommand{\pr}[1]{\ensuremath{\Pr\left(#1\right)}}
\providecommand{\prt}[2]{\ensuremath{p_{#1}^{\left(#2\right)} }}        % own macro for this question
\providecommand{\qfunc}[1]{\ensuremath{Q\left(#1\right)}}
\providecommand{\sbrak}[1]{\ensuremath{{}\left[#1\right]}}
\providecommand{\lsbrak}[1]{\ensuremath{{}\left[#1\right.}}
\providecommand{\rsbrak}[1]{\ensuremath{{}\left.#1\right]}}
\providecommand{\brak}[1]{\ensuremath{\left(#1\right)}}
\providecommand{\lbrak}[1]{\ensuremath{\left(#1\right.}}
\providecommand{\rbrak}[1]{\ensuremath{\left.#1\right)}}
\providecommand{\cbrak}[1]{\ensuremath{\left\{#1\right\}}}
\providecommand{\lcbrak}[1]{\ensuremath{\left\{#1\right.}}
\providecommand{\rcbrak}[1]{\ensuremath{\left.#1\right\}}}
\newcommand{\sgn}{\mathop{\mathrm{sgn}}}
\providecommand{\abs}[1]{\left\vert#1\right\vert}
\providecommand{\res}[1]{\Res\displaylimits_{#1}} 
\providecommand{\norm}[1]{\left\lVert#1\right\rVert}
%\providecommand{\norm}[1]{\lVert#1\rVert}
\providecommand{\mtx}[1]{\mathbf{#1}}
\providecommand{\mean}[1]{E\left[ #1 \right]}
\providecommand{\cond}[2]{#1\middle|#2}
\providecommand{\fourier}{\overset{\mathcal{F}}{ \rightleftharpoons}}
\newenvironment{amatrix}[1]{%
  \left(\begin{array}{@{}*{#1}{c}|c@{}}
}{%
  \end{array}\right)
}

\newcommand{\solution}{\noindent \textbf{Solution: }}
\newcommand{\cosec}{\,\text{cosec}\,}
\providecommand{\dec}[2]{\ensuremath{\overset{#1}{\underset{#2}{\gtrless}}}}
\newcommand{\myvec}[1]{\ensuremath{\begin{pmatrix}#1\end{pmatrix}}}
\newcommand{\mydet}[1]{\ensuremath{\begin{vmatrix}#1\end{vmatrix}}}
\newcommand{\myaugvec}[2]{\ensuremath{\begin{amatrix}{#1}#2\end{amatrix}}}
\providecommand{\rank}{\text{rank}}
\providecommand{\pr}[1]{\ensuremath{\Pr\left(#1\right)}}
\providecommand{\qfunc}[1]{\ensuremath{Q\left(#1\right)}}
	\newcommand*{\permcomb}[4][0mu]{{{}^{#3}\mkern#1#2_{#4}}}
\newcommand*{\perm}[1][-3mu]{\permcomb[#1]{P}}
\newcommand*{\comb}[1][-1mu]{\permcomb[#1]{C}}
\providecommand{\qfunc}[1]{\ensuremath{Q\left(#1\right)}}
\providecommand{\gauss}[2]{\mathcal{N}\ensuremath{\left(#1,#2\right)}}
\providecommand{\diff}[2]{\ensuremath{\frac{d{#1}}{d{#2}}}}
\providecommand{\myceil}[1]{\left \lceil #1 \right \rceil }
\newcommand\figref{Fig.~\ref}
\newcommand\tabref{Table~\ref}
\newcommand{\sinc}{\,\text{sinc}\,}
\newcommand{\rect}{\,\text{rect}\,}
\let\vec\mathbf
\vspace{3cm}
\title{
  Question 9.3.2
}
\author{EE22BTECH11051 - Sreekar Cheela}
\maketitle
There are 5\% defective items in a large bulk of items. What is the probability that a sample of 10 items will include not more than one defective item?\\
\solution\\
\fi
\begin{table}[h!]
    \begin{center}
       \begin{tabular}{|l|c|r|}
       \hline
       Parameter & Values & Description\\
       \hline
       $n$ & 10 & Number of articles\\
       \hline
       $p$ & 0.05 & Probability of being defective\\
       \hline
       $Y$ & $0\leq Y \leq 10$
       &  Number of defective elements\\
       \hline
       $\mu=np$ & $0.5$ & mean\\
       \hline
       $\sigma=\sqrt{np(1-p)}$ & $0.475$ & standard deviation\\
       \hline
       \end{tabular}
       \end{center}
   \end{table}
\text{Gaussian Distribution}\\
\begin{enumerate}
\item Central limit theorm:
\begin{align}
Y &\sim \gauss{\mu}{\frac{\sigma}{\sqrt{n}}}\\
\end{align}
Due to continuity correction \pr{X=x} can be approximated using gaussian distribution as
\begin{align}
	p_Y\brak{x}&\approx\pr{x-0.05<Y<x+0.05}\\
	&\approx\pr{Y<x+0.05}-\pr{Y<x-0.05}	\\
	&\approx F_Y\brak{x+0.05}-F_Y\brak{x-0.05}
\end{align}
Now, the CDF of Y can be found by;
\begin{align}
F_{Y}\brak{y} &= \pr{Y \leq y}\\
&= p_{Y}\brak{y}
\end{align}
We also know that;
\begin{align}
\qfunc{x} &= \pr{X > x}, x > 0, X \sim N(0,1)\\
\qfunc{-x} &= \pr{X > -x}, x < 0, X \sim N(0,1)\\
&= 1 - \qfunc{x}
\end{align}
Hence, the CDF is given as:
\begin{align}
F_Y\brak{y} &= \begin{cases}
                 1 - \qfunc{\frac{y - \mu}{\sigma}},& \text{if $y > \mu$}\\
                 1 - \qfunc{\frac{y - \mu}{\sigma}} = \qfunc{\frac{\mu - y}{\sigma}},& \text{if $y < \mu$}
               \end{cases}
\end{align}
Now, we get:
\begin{align}
F_Y\brak{1} &= p_{Y}\brak{1.05}\\
&= 1 - \qfunc{\frac{1.05 - 0.5}{\sqrt{0.05}}}\\
&= 1-\qfunc{\frac{0.55}{0.2236}}\\
&= 1-\qfunc{2.4596}\\
&= 0.99304
\end{align}
\item Binomial Distribution:
\begin{align}
n=10 ; p=\frac{1}{20}
\end{align}
Pmf of $X$ for $0 \leq k \leq 10$ is
\begin{align}
p_X(k)&=\comb{n}{k}p^k(1-p)^{n-k}
\end{align}
Then the probability is given as:
\begin{align}
p_X(0)+p_X(1)=\comb{10}{0}\left(\frac{1}{20}\right)^0\left(1-\frac{1}{20}\right)^{10}
+\comb{10}{1}\left(\frac{1}{20}\right)^1\left(1-\frac{1}{20}\right)^{9}
\end{align}
Hence we get;
\begin{align}
p_X(0)+p_X(1)=29\left(\frac{19^9}{20^{10}}\right)
            &=0.91386
\end{align}
Hence we can say probability calculated through central limit theorem is very close to 
the one calculated through binomial distribution.
\end{enumerate}
\begin{figure}[H]
\centering
\includegraphics[width=\columnwidth]{ncert/9/3/2/figs/figure1.png}
\caption{Binomial vs Gaussian}
\label{fig:9.3.2_gauss}
\end{figure}
