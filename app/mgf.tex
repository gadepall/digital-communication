\begin{enumerate}[label=\thechapter.\arabic*,ref=\thechapter.\theenumi]
\numberwithin{equation}{enumi}
\numberwithin{figure}{enumi}
\numberwithin{table}{enumi}
\item 
The $Z$-transform of $X$ is defined as
\begin{align}
M_X(z) = E\sbrak{z^{-X}} = \sum_{k=-\infty}^{\infty}p_X(k)z^{-k}
\label{eq:dice_xz}
\end{align}
\item If $X_1$ and $X_2$ are independent, the MGF of 
\begin{align}
	X = X_1 + X_2
\end{align}
is given by 
	\begin{align}
%p_X(n) &= p_{X_1}(n)*p_{X_2}(n),
%\\
M_X(z) &= P_{X_1}(z)P_{X_2}(z)
\label{eq:dice_xzprod_def}
\end{align}
The above property follows from Fourier analysis and is fundamental to signal processing. 
\item Let 
$X_i$ be independent.  For 
\begin{align}
X &= X_1+X_2+...+X_n,
\\
M_X(z)&=\prod_{i=1}^{n}M_{X_i}(z)
\end{align}

\item The $n$th moment of $X$ can be expressed as
\begin{align}
\label{eq:mgf-mean}
E\sbrak{X^n}&= \frac{d^nM_{X}(z^{-1})}{dz^n}\vert_{z=1}
\end{align}

\end{enumerate}
