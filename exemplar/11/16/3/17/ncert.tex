\iffalse
\let\negmedspace\undefined
\let\negthickspace\undefined
\documentclass[journal,12pt,twocolumn]{IEEEtran}
\usepackage{cite}
\usepackage{amsmath,amssymb,amsfonts,amsthm}
\usepackage{algorithmic}
\usepackage{graphicx}
\usepackage{textcomp}
\usepackage{xcolor}
\usepackage{txfonts}
\usepackage{listings}
\usepackage{enumitem}
\usepackage{mathtools}
\usepackage{gensymb}
\usepackage{comment}
\usepackage[breaklinks=true]{hyperref}
\usepackage{tkz-euclide} 
\usepackage{listings}
\usepackage{gvv}                                        
\def\inputGnumericTable{}                                 
\usepackage[latin1]{inputenc}                                
\usepackage{color}                                            
\usepackage{array}                                            
\usepackage{longtable}                                       
\usepackage{calc}                                             
\usepackage{multirow}                                         
\usepackage{hhline}                                           
\usepackage{ifthen}                                           
\usepackage{lscape}

\newtheorem{theorem}{Theorem}[section]
\newtheorem{problem}{Problem}
\newtheorem{proposition}{Proposition}[section]
\newtheorem{lemma}{Lemma}[section]
\newtheorem{corollary}[theorem]{Corollary}
\newtheorem{example}{Example}[section]
\newtheorem{definition}[problem]{Definition}
\newcommand{\BEQA}{\begin{eqnarray}}
\newcommand{\EEQA}{\end{eqnarray}}
\newcommand{\define}{\stackrel{\triangle}{=}}
\theoremstyle{remark}
\newtheorem{rem}{Remark}
\begin{document}

\bibliographystyle{IEEEtran}
\vspace{3cm}

\title{Probability Assignment}
\author{EE22BTECH11022-G.SAI HARSHITH$^{*}$% <-this % stops a space
}
\maketitle
\newpage
\bigskip
\renewcommand{\thefigure}{\theenumi}
\renewcommand{\thetable}{\theenumi}

Determine the probability $p$,for each of following events.
\begin{enumerate}
\item An odd number appears in a single roll of dice.
\item Atleast one head appears in two tosses of fair coin.
\item A king,9 of hearts or 3 of spades appears in drawing a single card from a well shuffled deck of 52 cards.
\item The sum of 6 appears in single toss of a pair of fair dice.
\end{enumerate}
\fi
\solution
\begin{enumerate}
\item Let the random variable $X$ be defined as:
\begin{table}[!ht]
	\input{exemplar/11/16/3/17/tables/randomvar1.tex}
\end{table}
   \begin{align}
        p_X(k) &= 
        \begin{cases}
            \frac{1}{6} & \text{if }k \in \{1, 2, 3, 4, 5, 6\}\\
            0 & \text{otherwise}
        \end{cases}\label{eq:11.16.3.17,1}
    \end{align}
 Let E be event occuring odd number on single roll. Since, the dice rolls are mutually exclusive. From \eqref{eq:11.16.3.17,1}.
 \begin{align}
 \pr{E}&=p_X(1)+p_X(3)+p_X(5)\\
 &=\frac{1}{6}+\frac{1}{6}+\frac{1}{6}\\
 &=\frac{1}{2} 
 \end{align}
 \item Let 1 be Head and 0 be Tail. Consider random variable $X_i$ where $i \in \{1,2\}$ as
 \begin{table}[!ht]
	\input{exemplar/11/16/3/17/tables/randomvar2.tex}
\end{table}\\
Let $Y=X_1+X_2$ be a binomial distribution with parameters
\begin{align}
n=2 \qquad p=\frac{1}{2}
\end{align}
Probabilty of getting k Head in $n=2$ coins is
\begin{align}
p_Y(k)&=\comb{2}{k}p^k(1-p)^{2-k}
\end{align}
Now, Let F be event of getting alteast one head.
\begin{align}
\pr{F}&=p_Y(1)+p_Y(2)\\
&=\comb{2}{1}\left(\frac{1}{2}\right)^1\left(1-\frac{1}{2}\right)^{2-1}+\comb{2}{2}\left(\frac{1}{2}\right)^2\left(1-\frac{1}{2}\right)^{2-2}\\
&=\frac{3}{4}
\end{align}
\item Let the random variables X and Y be defined as:
\begin{table}[!ht]
	\input{exemplar/11/16/3/17/tables/randomvar3.tex}
\end{table}\\
Let For $X \in \cbrak{1,2,3,4}$ represents Diamonds, Clubs, Hearts, Spades respectively.
\begin{align}
p_{X}(k) &= 
        \begin{cases}
            \frac{1}{4} & \text{if }1 \leq k \leq 4\\
            0 & \text{otherwise}
        \end{cases}\label{eq:11.16.3.17,11}\\
        p_{Y}(k) &= 
        \begin{cases}
            \frac{1}{13} & \text{if }1 \leq k \leq 13 \\
            0 & \text{otherwise}
        \end{cases}\label{eq:11.16.3.17,12}\\
p_{XY}(k,m) &= 
        \begin{cases}
            \frac{1}{52} & \text{if }1 \leq k \leq 4 \text{ and }1 \leq m \leq 13 \\
            0 & \text{otherwise}
        \end{cases}\label{eq:11.16.3.17,13}
    \end{align}
Let $Y=13$ represent king Card. So, Let G be event to get 4 kings, 9 of hearts,3 of spades. From \eqref{eq:11.16.3.17,12} and \eqref{eq:11.16.3.17,13}.
\begin{align}
\pr{G}&=p_{Y}(13)+p_{XY}(3,9)+p_{XY}(4,3)\\
&=\frac{1}{13}+\frac{1}{52}+\frac{1}{52}\\
&=\frac{3}{26}
\end{align}
\item  Let random variables $X_i$ where $i \in \{1,2\}$ be defined as
\begin{table}[!ht]
	\input{exemplar/11/16/3/17/tables/randomvar4.tex}
\end{table}\\
Let $Y=X_1+X_2$. Then 
\begin{align}
  p_Y(k)&=\begin{cases}
            \frac{k-1}{36} & \text{if }1 \leq k \leq 7\\
            \frac{13-k}{36} & \text{if }8 \leq k \leq 13\\
            0 & \text{otherwise}
        \end{cases}     
    \end{align}
Consider an H for which sum of both dice is six. Since the dice are independent.
\begin{align}
\pr{H}&=p_Y(6)\\
&=\frac{5}{36}
\end{align} 
    \end{enumerate}
