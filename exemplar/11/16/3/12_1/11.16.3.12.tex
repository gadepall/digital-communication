\iffalse
\let\negmedspace\undefined
\let\negthickspace\undefined
\documentclass[journal,12pt,twocolumn]{IEEEtran}
\usepackage{cite}
\usepackage{amsmath,amssymb,amsfonts,amsthm}
\usepackage{algorithmic}
\usepackage{graphicx}
\usepackage{textcomp}
\usepackage{xcolor}
\usepackage{txfonts}
\usepackage{listings}
%\usepackage{enumitem}
\usepackage{mathtools}
\usepackage{gensymb}
\usepackage[breaklinks=true]{hyperref}
\usepackage{tkz-euclide} % loads  TikZ and tkz-base
\usepackage{listings}
\usepackage[inline]{enumitem}
\DeclareMathOperator*{\Res}{Res}
\renewcommand\thesection{\arabic{section}}
\renewcommand\thesubsection{\thesection.\arabic{subsection}}
\renewcommand\thesubsubsection{\thesubsection.\arabic{subsubsection}}


\def\inputGnumericTable{}

\usepackage[latin1]{inputenc}                                 
\usepackage{color}                                            
\usepackage{array}                                            
\usepackage{longtable}                                        
\usepackage{calc}                                             
\usepackage{multirow}                                         
\usepackage{hhline}                                           
\usepackage{ifthen}
\usepackage{caption} 
\captionsetup[table]{skip=3pt}  
\providecommand{\pr}[1]{\ensuremath{\Pr\left(#1\right)}}
\providecommand{\cbrak}[1]{\ensuremath{\left\{#1\right\}}}

\renewcommand\thesectiondis{\arabic{section}}
\renewcommand\thesubsectiondis{\thesectiondis.\arabic{subsection}}
\renewcommand\thesubsubsectiondis{\thesubsectiondis.\arabic{subsubsection}}

\def\inputGnumericTable{}                                 %%

\lstset{
frame=single, 
breaklines=true,
columns=fullflexible
}

\begin{document}

\newtheorem{theorem}{Theorem}[section]
\newtheorem{problem}{Problem}
\newtheorem{proposition}{Proposition}[section]
\newtheorem{lemma}{Lemma}[section]
\newtheorem{corollary}[theorem]{Corollary}
\newtheorem{example}{Example}[section]
\newtheorem{definition}[problem]{Definition}
\newcommand{\BEQA}{\begin{eqnarray}}
\newcommand{\EEQA}{\end{eqnarray}}
\newcommand{\define}{\stackrel{\triangle}{=}}
\newcommand{\xor}{\oplus}
\bibliographystyle{IEEEtran}

\providecommand{\mbf}{\mathbf}
\providecommand{\pr}[1]{\ensuremath{\Pr\left(#1\right)}}
\providecommand{\qfunc}[1]{\ensuremath{Q\left(#1\right)}}
\providecommand{\sbrak}[1]{\ensuremath{{}\left[#1\right]}}
\providecommand{\lsbrak}[1]{\ensuremath{{}\left[#1\right.}}
\providecommand{\rsbrak}[1]{\ensuremath{{}\left.#1\right]}}
\providecommand{\brak}[1]{\ensuremath{\left(#1\right)}}
\providecommand{\lbrak}[1]{\ensuremath{\left(#1\right.}}
\providecommand{\rbrak}[1]{\ensuremath{\left.#1\right)}}
\providecommand{\cbrak}[1]{\ensuremath{\left\{#1\right\}}}
\providecommand{\lcbrak}[1]{\ensuremath{\left\{#1\right.}}
\providecommand{\rcbrak}[1]{\ensuremath{\left.#1\right\}}}
\theoremstyle{remark}
\newtheorem{rem}{Remark}
\newcommand{\sgn}{\mathop{\mathrm{sgn}}}

\newcommand{\solution}{\noindent \textbf{Solution: }}
\newcommand{\cosec}{\,\text{cosec}\,}
\providecommand{\dec}[2]{\ensuremath{\overset{#1}{\underset{#2}{\gtrless}}}}
\newcommand{\myvec}[1]{\ensuremath{\begin{pmatrix}#1\end{pmatrix}}}
\newcommand{\mydet}[1]{\ensuremath{\begin{vmatrix}#1\end{vmatrix}}}

\let\vec\mathbf


\vspace{3cm}

\title{
  

  Assignment -2 in \LaTeX
    
  }
  \author{ Muzaan Mohammed Faizel A P\\
  EE22BTECH11036
  }	
% make the title area
\maketitle
\newpage
\bigskip
\renewcommand{\thefigure}{\theenumi}
\renewcommand{\thetable}{\theenumi}
\renewcommand{\thetable}{\arabic{table}}  
\textbf{11.16.3.12}:
One urn contains two black balls (labelled B1 and B2) and one white ball. A
second urn contains one black ball and two white balls (labelled W1 and W2).
Suppose the following experiment is performed. One of the two urns is chosen
at random. Next a ball is randomly chosen from the urn. Then a second ball is
chosen at random from the same urn without replacing the first ball.

\begin{enumerate}[label=(\alph*)]
% \item Write the sample space showing all possible outcomes
\item What is the probability that two black balls are chosen?

\item What is the probability that two balls of opposite colour are chosen?
\end{enumerate}


%Answer
\solution
\fi
Let $X$ be a Bernoulli random variable
    \begin{align} 
    X=
    \begin{cases}
      0, & \text{ Urn 1  }\\
      1, & \text{Urn 2}
    \end{cases}
\end{align}
  Since both events are equally likely
  \begin{align}
  \pr{X=0} &= \pr{X=1}\\
         &=\frac{1}{2}
  \end{align}
  Let $Y_i$ be a random variable to denote the  turn
  \begin{align}
    Y_i=
    \begin{cases}
      0, & \text{Black ball }\\
      1, & \text{White ball }
    \end{cases}
  \end{align}
$Y_1$ denotes the first ball and $Y_2$ denotes the second ball.
\begin{table}[ht!]
		\centering
		%%%%%%%%%%%%%%%%%%%%%%%%%%%%%%%%%%%%%%%%%%%%%%%%%%%%%%%%%%%%%%%%%%%%%%
%%                                                                  %%
%%  This is a LaTeX2e table fragment exported from Gnumeric.        %%
%%                                                                  %%
%%%%%%%%%%%%%%%%%%%%%%%%%%%%%%%%%%%%%%%%%%%%%%%%%%%%%%%%%%%%%%%%%%%%%%
\begin{tabular}{|c|c|c|}
\hline
Random Variable  & Value of the random variable    & Event                            \\
\hline
\multirow{2}{*}B  & 0                            & selecting first bag              \\
\cline{2-3}
                 & 1                            & selecting second bag             \\
\hline
\multirow{2}{*}R  & 0                            & choosing white ball from the bag \\
\cline{2-3}
                 & 1                            & choosing red ball from the bag   \\
\hline
\end{tabular}

		\caption{}
		\label{tab:exemplar/11/16/3/12/table1}	
	\end{table}
\begin{enumerate}

\item 
\begin{align}
E &= \brak{Y_1+Y_2}^\prime  \\
&= Y_1^\prime Y_2^\prime
\end{align}
Required Probability:
\begin{align}  
&\pr{Y_1^\prime Y_2^\prime} \\ \nonumber
&=\pr{Y_1^\prime Y_2^\prime | X^\prime}\pr{X^\prime} \\ \nonumber
&= \frac{2}{3}\times\frac{1}{2}\times\frac{1}{2} \\ \label{eq:1}
&= 1/6 
\end{align}
Therefore,
\begin{align}
\pr{E} = \frac{1}{6} 
\end{align}
\item 
\begin{align}
E &= Y_1 \xor Y_2 \\
&= Y_1 Y_2^\prime + Y_1^\prime Y_2
\end{align}
Required Probability : 
\begin{align}
&\pr{Y_1 Y_2^\prime + Y_1^\prime Y_2} \\
&=\pr{Y_1 Y_2^\prime} + \pr{Y_1^\prime Y_2} \\
&=\brak{\frac{1}{3}\times\frac{1}{2} + \frac{2}{3}\times\frac{1}{2}\times\frac{1}{2}}\times 2 \\
&=\frac{2}{3}
\end{align}
Therefore,
\begin{align}
\pr{E} &= \frac{2}{3}
\end{align}
\end{enumerate}
