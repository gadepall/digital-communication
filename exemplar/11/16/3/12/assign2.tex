\let\negmedspace\undefined
\let\negthickspace\undefined
\documentclass[journal,12pt,twocolumn]{IEEEtran}
\usepackage{cite}
\usepackage{amsmath,amssymb,amsfonts,amsthm}
\usepackage{algorithmic}
\usepackage{graphicx}
\usepackage{textcomp}
\usepackage{xcolor}
\usepackage{txfonts}
\usepackage{listings}
%\usepackage{enumitem}
\usepackage{mathtools}
\usepackage{gensymb}
\usepackage[breaklinks=true]{hyperref}
\usepackage{tkz-euclide} % loads  TikZ and tkz-base
\usepackage{listings}
\usepackage[inline]{enumitem}
\DeclareMathOperator*{\Res}{Res}
\renewcommand\thesection{\arabic{section}}
\renewcommand\thesubsection{\thesection.\arabic{subsection}}
\renewcommand\thesubsubsection{\thesubsection.\arabic{subsubsection}}


\def\inputGnumericTable{}

\usepackage[latin1]{inputenc}                                 
\usepackage{color}                                            
\usepackage{array}                                            
\usepackage{longtable}                                        
\usepackage{calc}                                             
\usepackage{multirow}                                         
\usepackage{hhline}                                           
\usepackage{ifthen}
\usepackage{caption} 
\captionsetup[table]{skip=3pt}  
\providecommand{\pr}[1]{\ensuremath{\Pr\left(#1\right)}}
\providecommand{\cbrak}[1]{\ensuremath{\left\{#1\right\}}}

\renewcommand\thesectiondis{\arabic{section}}
\renewcommand\thesubsectiondis{\thesectiondis.\arabic{subsection}}
\renewcommand\thesubsubsectiondis{\thesubsectiondis.\arabic{subsubsection}}

\def\inputGnumericTable{}                                 %%

\lstset{
frame=single, 
breaklines=true,
columns=fullflexible
}

\begin{document}

\newtheorem{theorem}{Theorem}[section]
\newtheorem{problem}{Problem}
\newtheorem{proposition}{Proposition}[section]
\newtheorem{lemma}{Lemma}[section]
\newtheorem{corollary}[theorem]{Corollary}
\newtheorem{example}{Example}[section]
\newtheorem{definition}[problem]{Definition}
\newcommand{\BEQA}{\begin{eqnarray}}
\newcommand{\EEQA}{\end{eqnarray}}
\newcommand{\define}{\stackrel{\triangle}{=}}
\newcommand{\xor}{\oplus}
\bibliographystyle{IEEEtran}

\providecommand{\mbf}{\mathbf}
\providecommand{\pr}[1]{\ensuremath{\Pr\left(#1\right)}}
\providecommand{\qfunc}[1]{\ensuremath{Q\left(#1\right)}}
\providecommand{\sbrak}[1]{\ensuremath{{}\left[#1\right]}}
\providecommand{\lsbrak}[1]{\ensuremath{{}\left[#1\right.}}
\providecommand{\rsbrak}[1]{\ensuremath{{}\left.#1\right]}}
\providecommand{\brak}[1]{\ensuremath{\left(#1\right)}}
\providecommand{\lbrak}[1]{\ensuremath{\left(#1\right.}}
\providecommand{\rbrak}[1]{\ensuremath{\left.#1\right)}}
\providecommand{\cbrak}[1]{\ensuremath{\left\{#1\right\}}}
\providecommand{\lcbrak}[1]{\ensuremath{\left\{#1\right.}}
\providecommand{\rcbrak}[1]{\ensuremath{\left.#1\right\}}}
\theoremstyle{remark}
\newtheorem{rem}{Remark}
\newcommand{\sgn}{\mathop{\mathrm{sgn}}}

\newcommand{\solution}{\noindent \textbf{Solution: }}
\newcommand{\cosec}{\,\text{cosec}\,}
\providecommand{\dec}[2]{\ensuremath{\overset{#1}{\underset{#2}{\gtrless}}}}
\newcommand{\myvec}[1]{\ensuremath{\begin{pmatrix}#1\end{pmatrix}}}
\newcommand{\mydet}[1]{\ensuremath{\begin{vmatrix}#1\end{vmatrix}}}

\let\vec\mathbf


\vspace{3cm}

\title{
%	\logo{
   Assignment 2\\ \Large AI1110: Probability and Random Variables \\ \large Indian Institute of Technology Hyderabad
%	}
}
\author{ Aditya Garg \\ CS22BTECH11002 \\ 26 April 2023	
	
}	
% make the title area
\maketitle
\newpage
\bigskip
\renewcommand{\thefigure}{\theenumi}
\renewcommand{\thetable}{\theenumi}
\renewcommand{\thetable}{\arabic{table}}  
\textbf{11.16.3.12}:
One urn contains two black balls (labelled B1 and B2) and one white ball. A
second urn contains one black ball and two white balls (labelled W1 and W2).
Suppose the following experiment is performed. One of the two urns is chosen
at random. Next a ball is randomly chosen from the urn. Then a second ball is
chosen at random from the same urn without replacing the first ball.

\begin{enumerate}[label=(\alph*)]
% \item Write the sample space showing all possible outcomes
\item What is the probability that two black balls are chosen?

\item What is the probability that two balls of opposite colour are chosen?
\end{enumerate}


%Answer
\solution

% Probability of an event $E$, written as $\pr{E}$
% \begin{align}
% \pr{E}=\displaystyle\frac{\text{Number of outcomes favourable to $E$}}{\text{Total Number of possible outcomes }}
% \end{align}
Let $Z$ be a Bernoulli random variable
 \begin{equation}
    Z=
    \begin{cases}
      0, & \text{if Urn 1 chosen }\\
      1, & \text{if Urn 2 chosen}
    \end{cases}
  \end{equation}
  Since both events are equally likely
  \begin{align}
  \pr{Z=0} &= \pr{Z=1}\\
         &=\frac{1}{2}
  \end{align}
  Let $X_i$ be a random variable 
  where $i$ denotes the turn
  \begin{equation}
    X_i=
    \begin{cases}
      0, & \text{if Black ball chosen }\\
      1, & \text{if White ball chosen}
    \end{cases}
  \end{equation}
Let $X_1$ be a random variable denoting first ball is chosen and
 $X_2$ be random variable denoting second ball is chosen .
 % \begin{enumerate}
 %     \item $(X+Y =1,Z=0) $ represents 2 black balls are chosen.
 %     \item $(X+Y =1,Z=1) $ represents 2 white balls are chosen.
 %     \item $ (X+Y) > 1 $ represents 2 balls of opposite colors are chosen.
 % \end{enumerate}
 
 

\begin{table}[ht!]
		\centering
		%%%%%%%%%%%%%%%%%%%%%%%%%%%%%%%%%%%%%%%%%%%%%%%%%%%%%%%%%%%%%%%%%%%%%%
%%                                                                  %%
%%  This is a LaTeX2e table fragment exported from Gnumeric.        %%
%%                                                                  %%
%%%%%%%%%%%%%%%%%%%%%%%%%%%%%%%%%%%%%%%%%%%%%%%%%%%%%%%%%%%%%%%%%%%%%%

\begin{center}
\begin{tabular}{|c|c|c|}
\hline
\textbf{RV}& \textbf{Values} & \textbf{Description} \\ \hline
$X$		   & 	$\{0,1\}$	&  1st draw - 0: Red, 1: Black\\ \hline
$Y$ 		   & 	$\{0,1\}$	&  2nd draw - 0: Red, 1: Black\\ \hline
\end{tabular}
\end{center}

		\caption{}
		\label{table:table1}	
	\end{table}

% \begin{table}[ht!]
% 		\centering
% 		\input{table-2.tex}
% 		\caption{}
% 		\label{table:table2}	
% 	\end{table}

 
\begin{enumerate}[label =(\alph*)]
% \item Sample Space S:
% \begin{align}
% \lcbrak{001,010,002,020,021,012,101,}\\
%  \rcbrak{110,102,120,121,112} 
% \end{align} 
% \begin{align}
%     \therefore n(S) = 12
% % \end{align}
\item Let $E$ be event that 2 black balls are chosen.
\begin{align}
E &= \brak{X_1+X_2}^\prime  \\
&= X_1^\prime X_2^\prime
\end{align}
Required Probability:
\begin{align}  
&\pr{X_1^\prime X_2^\prime} \\ \nonumber
&=\pr{X_1^\prime X_2^\prime | Z^\prime}\pr{Z^\prime} \\ \nonumber
&= \frac{2}{3}\times\frac{1}{2}\times\frac{1}{2} \\ \label{eq:1}
&= 1/6 
\end{align}
\begin{align}
\therefore \pr{E} = \frac{1}{6} 
\end{align}
\item Let $E$ be event that balls of opposite colours are chosen.
\begin{align}
E &= X_1 \xor X_2 \\
&= X_1 X_2^\prime + X_1^\prime X_2
\end{align}
Required Probability : 
\begin{align}
&\pr{X_1 X_2^\prime + X_1^\prime X_2} \\
&=\pr{X_1 X_2^\prime} + \pr{X_1^\prime X_2} \\
&\because \pr{X_1 X_1 ^\prime X_2 X_2 ^\prime} = 0 \\
&=\brak{\frac{1}{3}\times\frac{1}{2} + \frac{2}{3}\times\frac{1}{2}\times\frac{1}{2}}\times 2 \\
&=\frac{2}{3}
\end{align}
\begin{align}
\therefore \pr{E} &= \frac{2}{3}
\end{align}
\end{enumerate}
\end{document}
