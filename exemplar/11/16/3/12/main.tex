\iffalse
\let\negmedspace\undefined
\let\negthickspace\undefined
\documentclass[journal,12pt,twocolumn]{IEEEtran}
\usepackage{cite}
\usepackage{amsmath,amssymb,amsfonts,amsthm}
\usepackage{algorithmic}
\usepackage{graphicx}
\usepackage{textcomp}
\usepackage{xcolor}
\usepackage{txfonts}
\usepackage{listings}
%\usepackage{enumitem}
\usepackage{mathtools}
\usepackage{gensymb}
\usepackage[breaklinks=true]{hyperref}
\usepackage{tkz-euclide} % loads  TikZ and tkz-base
\usepackage{listings}
\usepackage[inline]{enumitem}
\DeclareMathOperator*{\Res}{Res}
\renewcommand\thesection{\arabic{section}}
\renewcommand\thesubsection{\thesection.\arabic{subsection}}
\renewcommand\thesubsubsection{\thesubsection.\arabic{subsubsection}}


\def\inputGnumericTable{}

\usepackage[latin1]{inputenc}                                 
\usepackage{color}                                            
\usepackage{array}                                            
\usepackage{longtable}                                        
\usepackage{calc}                                             
\usepackage{multirow}                                         
\usepackage{hhline}                                           
\usepackage{ifthen}
\usepackage{caption} 
\captionsetup[table]{skip=3pt}  
\providecommand{\pr}[1]{\ensuremath{\Pr\left(#1\right)}}
\providecommand{\cbrak}[1]{\ensuremath{\left\{#1\right\}}}

\renewcommand\thesectiondis{\arabic{section}}
\renewcommand\thesubsectiondis{\thesectiondis.\arabic{subsection}}
\renewcommand\thesubsubsectiondis{\thesubsectiondis.\arabic{subsubsection}}

\def\inputGnumericTable{}                                 %%

\lstset{
frame=single, 
breaklines=true,
columns=fullflexible
}

\begin{document}

\newtheorem{theorem}{Theorem}[section]
\newtheorem{problem}{Problem}
\newtheorem{proposition}{Proposition}[section]
\newtheorem{lemma}{Lemma}[section]
\newtheorem{corollary}[theorem]{Corollary}
\newtheorem{example}{Example}[section]
\newtheorem{definition}[problem]{Definition}
\newcommand{\BEQA}{\begin{eqnarray}}
\newcommand{\EEQA}{\end{eqnarray}}
\newcommand{\define}{\stackrel{\triangle}{=}}
\newcommand{\xor}{\oplus}
\bibliographystyle{IEEEtran}

\providecommand{\mbf}{\mathbf}
\providecommand{\pr}[1]{\ensuremath{\Pr\left(#1\right)}}
\providecommand{\qfunc}[1]{\ensuremath{Q\left(#1\right)}}
\providecommand{\sbrak}[1]{\ensuremath{{}\left[#1\right]}}
\providecommand{\lsbrak}[1]{\ensuremath{{}\left[#1\right.}}
\providecommand{\rsbrak}[1]{\ensuremath{{}\left.#1\right]}}
\providecommand{\brak}[1]{\ensuremath{\left(#1\right)}}
\providecommand{\lbrak}[1]{\ensuremath{\left(#1\right.}}
\providecommand{\rbrak}[1]{\ensuremath{\left.#1\right)}}
\providecommand{\cbrak}[1]{\ensuremath{\left\{#1\right\}}}
\providecommand{\lcbrak}[1]{\ensuremath{\left\{#1\right.}}
\providecommand{\rcbrak}[1]{\ensuremath{\left.#1\right\}}}
\theoremstyle{remark}
\newtheorem{rem}{Remark}
\newcommand{\sgn}{\mathop{\mathrm{sgn}}}

\newcommand{\solution}{\noindent \textbf{Solution: }}
\newcommand{\cosec}{\,\text{cosec}\,}
\providecommand{\dec}[2]{\ensuremath{\overset{#1}{\underset{#2}{\gtrless}}}}
\newcommand{\myvec}[1]{\ensuremath{\begin{pmatrix}#1\end{pmatrix}}}
\newcommand{\mydet}[1]{\ensuremath{\begin{vmatrix}#1\end{vmatrix}}}

\let\vec\mathbf


\vspace{3cm}

\title{
%	\logo{
   Assignment 2\\ \Large AI1110: Probability and Random Variables \\ \large Indian Institute of Technology Hyderabad
%	}
}
\author{ Aditya Garg \\ CS22BTECH11002 \\ 26 April 2023	
	
}	
% make the title area
\maketitle
\newpage
\bigskip
\renewcommand{\thefigure}{\theenumi}
\renewcommand{\thetable}{\theenumi}
\renewcommand{\thetable}{\arabic{table}}  
\textbf{11.16.3.12}:


%Answer
\solution

% Probability of an event $E$, written as $\pr{E}$
% \begin{align}
% \pr{E}=\displaystyle\frac{\text{Number of outcomes favourable to $E$}}{\text{Total Number of possible outcomes }}
% \end{align}
  and 
  \begin{equation}
    X_i=
    \begin{cases}
      0, & \text{if Black ball chosen }\\
      1, & \text{if White ball chosen}
    \end{cases}
  \end{equation}
\fi
Let 
 \begin{equation}
    Z=
    \begin{cases}
      0, & \text{if Urn 1 chosen }\\
      1, & \text{if Urn 2 chosen}
    \end{cases}
  \end{equation}
Let $X_1$ be a random variable denoting first ball is chosen and
 $X_2$ be random variable denoting second ball is chosen .
See 
		\tabref{tab:exemplar/11/16/3/12/table1}	
\begin{table}[ht!]
		\centering
		%%%%%%%%%%%%%%%%%%%%%%%%%%%%%%%%%%%%%%%%%%%%%%%%%%%%%%%%%%%%%%%%%%%%%%
%%                                                                  %%
%%  This is a LaTeX2e table fragment exported from Gnumeric.        %%
%%                                                                  %%
%%%%%%%%%%%%%%%%%%%%%%%%%%%%%%%%%%%%%%%%%%%%%%%%%%%%%%%%%%%%%%%%%%%%%%

\begin{center}
\begin{tabular}{|c|c|c|}
\hline
\textbf{RV}& \textbf{Values} & \textbf{Description} \\ \hline
$X$		   & 	$\{0,1\}$	&  1st draw - 0: Red, 1: Black\\ \hline
$Y$ 		   & 	$\{0,1\}$	&  2nd draw - 0: Red, 1: Black\\ \hline
\end{tabular}
\end{center}

		\caption{}
		\label{tab:exemplar/11/16/3/12/table1}	
	\end{table}
%
\begin{enumerate}
\item Let $E$ be event that 2 black balls are chosen.
\begin{align}
E &= \brak{X_1+X_2}^\prime  
= X_1^\prime X_2^\prime
\end{align}
The desired probability is
\begin{align}  
\pr{X_1^\prime X_2^\prime} 
&=\pr{X_1^\prime X_2^\prime | Z^\prime}\pr{Z^\prime} \\
&= \frac{2}{3}\times\frac{1}{2}\times\frac{1}{2} 
= 1/6 
\label{eq:exemplar/11/16/3/12/1}
\\
\implies \pr{E} = \frac{1}{6} 
\end{align}
\item Let $F$ be event that balls of opposite colours are chosen.
\begin{align}
F &= X_1 \xor X_2 
= X_1 X_2^\prime + X_1^\prime X_2
\end{align}
The desired probability is 
\begin{align}
\pr{X_1 X_2^\prime + X_1^\prime X_2} 
&=\pr{X_1 X_2^\prime} + \pr{X_1^\prime X_2} \quad
\because \pr{X_1 X_1 ^\prime X_2 X_2 ^\prime} = 0 \\
&=\brak{\frac{1}{3}\times\frac{1}{2} + \frac{2}{3}\times\frac{1}{2}\times\frac{1}{2}}\times 2 
=\frac{2}{3}
\\
\implies \pr{F} &= \frac{2}{3}
\end{align}
\end{enumerate}
