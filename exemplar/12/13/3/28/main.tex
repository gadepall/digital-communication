\iffalse
\let\negmedspace\undefined
\let\negthickspace\undefined
\documentclass[journal,12pt,twocolumn]{IEEEtran}
\usepackage{cite}
\usepackage{amsmath,amssymb,amsfonts,amsthm}
\usepackage{algorithmic}
\usepackage{graphicx}
\usepackage{textcomp}
\usepackage{xcolor}
\usepackage{txfonts}
\usepackage{listings}
\usepackage{enumitem}
\usepackage{mathtools}
\usepackage{gensymb}
\usepackage{comment}
\usepackage[breaklinks=true]{hyperref}
\usepackage{tkz-euclide} 
\usepackage{listings}
\usepackage{gvv}                                        
\def\inputGnumericTable{}                                 
\usepackage[latin1]{inputenc}                                
\usepackage{color}                                            
\usepackage{array}                                            
\usepackage{longtable}                                       
\usepackage{calc}                                             
\usepackage{multirow}                                         
\usepackage{hhline}                                           
\usepackage{ifthen}                                           
\usepackage{lscape}

\newtheorem{theorem}{Theorem}[section]
\newtheorem{problem}{Problem}
\newtheorem{proposition}{Proposition}[section]
\newtheorem{lemma}{Lemma}[section]
\newtheorem{corollary}[theorem]{Corollary}
\newtheorem{example}{Example}[section]
\newtheorem{definition}[problem]{Definition}
\newcommand{\BEQA}{\begin{eqnarray}}
\newcommand{\EEQA}{\end{eqnarray}}
\newcommand{\define}{\stackrel{\triangle}{=}}
\theoremstyle{remark}
\newtheorem{rem}{Remark}
\begin{document}

\bibliographystyle{IEEEtran}
\vspace{3cm}

\title{Exemplar - 12.13.3.28}
\author{EE22BTECH11039 - Pandrangi Aditya Sriram$^{*}$% <-this % stops a space
}
\maketitle
\newpage
\bigskip

\renewcommand{\thefigure}{\theenumi}
\renewcommand{\thetable}{\theenumi}


\vspace{3cm}
\textbf{Question:} A die is thrown three times. Let X be 'the number of twos seen'. Find the expectation of X.\\
\solution
\fi
Let the random variables be:
\begin{table}[h!]
    \input{exemplar/12/13/3/28/tables/randomvar.tex}
    \caption{Random Variables}
    \label{tab:12_13_3_28}
\end{table}\\
For a single die roll, since the probability of rolling a two is $\frac{1}{6}$ the probability distribution function of $X_i$ is:
\begin{align}
    p_{X_i}(k) =
    \begin{cases}
        \frac{1}{6} & \text{if } k = 1\\
        \frac{5}{6} & \text{if } k = 0
    \end{cases}
\end{align}\\
Thus, 
\begin{align}
    E\brak{X_1} = E\brak{X_2} = E\brak{X_3} &= \sum_{k = 0}^{1} k p_{X_i}(k) \\
    &= \frac{5}{6}\brak{0} + \frac{1}{6}\brak{1}\\
    &= \frac{1}{6}
\end{align}
But, as all three dice rolls are independent, and expectation is linear:
\begin{align}
    X &= X_1 + X_2 + X_3\\
    \therefore E\brak{X} &= E\brak{X_1 + X_2 + X_3}\\
    &= E\brak{X_1} + E\brak{X_2} + E\brak{X_3} \\
    &= \frac{1}{2}
\end{align}
