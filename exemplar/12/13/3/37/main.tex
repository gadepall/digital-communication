\iffalse
\let\negmedspace\undefined
\let\negthickspace\undefined
\documentclass[journal,12pt,twocolumn]{IEEEtran}
\usepackage{cite}
\usepackage{amsmath,amssymb,amsfonts,amsthm}
\usepackage{algorithmic}
\usepackage{graphicx}
\usepackage{textcomp}
\usepackage{xcolor}
\usepackage{txfonts}
\usepackage{listings}
\usepackage{enumitem}
\usepackage{mathtools}
\usepackage{gensymb}
\usepackage{comment}
\usepackage[breaklinks=true]{hyperref}
\usepackage{tkz-euclide} 
\usepackage{listings}
\usepackage{gvv}                                        
\def\inputGnumericTable{}                                 
\usepackage[latin1]{inputenc}                                
\usepackage{color}                                            
\usepackage{array}                                            
\usepackage{longtable}                                       
\usepackage{calc}                                             
\usepackage{multirow}                                         
\usepackage{hhline}                                           
\usepackage{ifthen}                                           
\usepackage{lscape}

\newtheorem{theorem}{Theorem}[section]
\newtheorem{problem}{Problem}
\newtheorem{proposition}{Proposition}[section]
\newtheorem{lemma}{Lemma}[section]
\newtheorem{corollary}[theorem]{Corollary}
\newtheorem{example}{Example}[section]
\newtheorem{definition}[problem]{Definition}
\newcommand{\BEQA}{\begin{eqnarray}}
\newcommand{\EEQA}{\end{eqnarray}}
\newcommand{\define}{\stackrel{\triangle}{=}}
\theoremstyle{remark}
\newtheorem{rem}{Remark}
\begin{document}

\bibliographystyle{IEEEtran}
\vspace{3cm}

\title{Probability Assignment}
\author{EE22BTECH11022-G.SAI HARSHITH$^{*}$% <-this % stops a space
}
\maketitle
\newpage
\bigskip
\renewcommand{\thefigure}{\theenumi}
\renewcommand{\thetable}{\theenumi}

Question: Find the variance of distribution.
\begin{table}[!ht]
	
%%%%%%%%%%%%%%%%%%%%%%%%%%%%%%%%%%%%%%%%%%%%%%%%%%%%%%%%%%%%%%%%%%%%%%
%%                                                                  %%
%%  This is a LaTeX2e table fragment exported from Gnumeric.        %%
%%                                                                  %%
%%%%%%%%%%%%%%%%%%%%%%%%%%%%%%%%%%%%%%%%%%%%%%%%%%%%%%%%%%%%%%%%%%%%%%

\begin{center}
\begin{tabular}{|c|c|l|}
\hline
 Parameter&  Value &Description\\ \hline
 $X$ & $bin(n,p)$ & no of correct answers  that candidate gets by guessing\\\hline
$n$ &  5 & total no of questions\\ \hline
 $p$ &  $\frac{1}{3}$ & probability of getting correct answer by guessing\\\hline

\end{tabular}
\end{center}
\end{table}\\
\fi
\solution
Calculating $E(X)$.
\begin{align}
E(X)&=\sum_{k=0}^{5} kp_X(k)\\
&=0\left(\frac{1}{6}\right)+1\left(\frac{5}{18}\right)+2\left(\frac{2}{9}\right)+3\left(\frac{1}{6}\right)+4\left(\frac{1}{9}\right)+5\left(\frac{1}{18}\right)\\
&=\frac{35}{18}
\label{eq:12.13.3.37,4}
\end{align}
Calculating $E(X^2)$
\begin{align}
E(X^2)&=\sum_{k=0}^{5} k^2p_X(k)\\
&=0^2\left(\frac{1}{6}\right)+1^2\left(\frac{5}{18}\right)+2^2\left(\frac{2}{9}\right)+3^2\left(\frac{1}{6}\right)+4^2\left(\frac{1}{9}\right)+5^2\left(\frac{1}{18}\right)\\
&=\frac{105}{18}
\label{eq:12.13.3.37,7}
\end{align}
From \eqref{eq:12.13.3.37,4} and \eqref{eq:12.13.3.37,7}.
\begin{align}
\sigma^2&=E(X^2)-[E(X)]^2\\
&=\frac{105}{18}-\left(\frac{35}{18}\right)^2\\
&=\frac{665}{324}
\end{align}
