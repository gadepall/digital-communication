\iffalse
\let\negmedspace\undefined
\let\negthickspace\undefined
\documentclass[article]{IEEEtran}
       \def\inputGnumericTable{}                                 %%
\usepackage{cite}
\usepackage{amsmath,amssymb,amsfonts,amsthm}
\usepackage{algorithmic}
\usepackage{graphicx}
\usepackage{textcomp}
\usepackage{xcolor}
\usepackage{txfonts}
\usepackage{listings}
\usepackage{enumitem}
\usepackage{mathtools}
\usepackage{gensymb}
\usepackage[breaklinks=true]{hyperref}
\usepackage{tkz-euclide} % loads  TikZ and tkz-base
\usepackage{listings}
\renewcommand{\theenumi}{\Alph{enumi}}
%
%\usepackage{setspace}
%\usepackage{gensymb}
%\doublespacing
%\singlespacing

%\usepackage{graphicx}
%\usepackage{amssymb}
%\usepackage{relsize}
%\usepackage[cmex10]{amsmath}
%\usepackage{amsthm}
%\interdisplaylinepenalty=2500
%\savesymbol{iint}
%\usepackage{txfonts}
%\restoresymbol{TXF}{iint}
%\usepackage{wasysym}
%\usepackage{amsthm}
%\usepackage{iithtlc}
%\usepackage{mathrsfs}
%\usepackage{txfonts}
%\usepackage{stfloats}
%\usepackage{bm}
%\usepackage{cite}
%\usepackage{cases}
%\usepackage{subfig}
%\usepackage{xtab}
%\usepackage{longtable}
%\usepackage{multirow}
%\usepackage{algorithm}
%\usepackage{algpseudocode}
%\usepackage{enumitem}
%\usepackage{mathtools}
%\usepackage{tikz}
%\usepackage{circuitikz}
%\usepackage{verbatim}
%\usepackage{tfrupee}
%\usepackage{stmaryrd}
%\usetkzobj{all}
    \usepackage{color}                                            %%
    \usepackage{array}                                            %%
    \usepackage{longtable}                                        %%
    \usepackage{calc}                                             %%
    \usepackage{multirow}                                         %%
    \usepackage{hhline}                                           %%
    \usepackage{ifthen}                                           %%
 %optionally (for landscape tables embedded in another document): %%
    \usepackage{lscape}     
%\usepackage{multicol}
%\usepackage{chngcntr}
%\usepackage{enumerate}

%\usepackage{wasysym}
%\documentclass[conference]{IEEEtran}
%\IEEEoverridecommandlockouts
% The preceding line is only needed to identify funding in the first footnote. If that is unneeded, please comment it out.

\newtheorem{theorem}{Theorem}[section]
\newtheorem{problem}{Problem}
\newtheorem{proposition}{Proposition}[section]
\newtheorem{lemma}{Lemma}[section]
\newtheorem{corollary}[theorem]{Corollary}
\newtheorem{example}{Example}[section]
\newtheorem{definition}[problem]{Definition}
%\newtheorem{thm}{Theorem}[section] 
%\newtheorem{defn}[thm]{Definition}
%\newtheorem{algorithm}{Algorithm}[section]
%\newtheorem{cor}{Corollary}
\newcommand{\BEQA}{\begin{eqnarray}}
\newcommand{\EEQA}{\end{eqnarray}}
\newcommand{\define}{\stackrel{\triangle}{=}}
\theoremstyle{remark}
\newtheorem{rem}{Remark}

\begin{document}
\providecommand{\pr}[1]{\ensuremath{\Pr\left(#1\right)}}
\providecommand{\prt}[2]{\ensuremath{p_{#1}^{\left(#2\right)} }}        % own macro for this question
\providecommand{\qfunc}[1]{\ensuremath{Q\left(#1\right)}}
\providecommand{\sbrak}[1]{\ensuremath{{}\left[#1\right]}}
\providecommand{\lsbrak}[1]{\ensuremath{{}\left[#1\right.}}
\providecommand{\rsbrak}[1]{\ensuremath{{}\left.#1\right]}}
\providecommand{\brak}[1]{\ensuremath{\left(#1\right)}}
\providecommand{\lbrak}[1]{\ensuremath{\left(#1\right.}}
\providecommand{\rbrak}[1]{\ensuremath{\left.#1\right)}}
\providecommand{\cbrak}[1]{\ensuremath{\left\{#1\right\}}}
\providecommand{\lcbrak}[1]{\ensuremath{\left\{#1\right.}}
\providecommand{\rcbrak}[1]{\ensuremath{\left.#1\right\}}}
\newcommand{\sgn}{\mathop{\mathrm{sgn}}}
\providecommand{\abs}[1]{\left\vert#1\right\vert}
\providecommand{\res}[1]{\Res\displaylimits_{#1}} 
\providecommand{\norm}[1]{\left\lVert#1\right\rVert}
%\providecommand{\norm}[1]{\lVert#1\rVert}
\providecommand{\mtx}[1]{\mathbf{#1}}
\providecommand{\mean}[1]{E\left[ #1 \right]}
\providecommand{\cond}[2]{#1\middle|#2}
\providecommand{\fourier}{\overset{\mathcal{F}}{ \rightleftharpoons}}
\newenvironment{amatrix}[1]{%
  \left(\begin{array}{@{}*{#1}{c}|c@{}}
}{%
  \end{array}\right)
}
%\providecommand{\hilbert}{\overset{\mathcal{H}}{ \rightleftharpoons}}
%\providecommand{\system}{\overset{\mathcal{H}}{ \longleftrightarrow}}
	%\newcommand{\solution}[2]{\textbf{Solution:}{#1}}
\newcommand{\solution}{\noindent \textbf{Solution: }}
\newcommand{\cosec}{\,\text{cosec}\,}
\providecommand{\dec}[2]{\ensuremath{\overset{#1}{\underset{#2}{\gtrless}}}}
\newcommand{\myvec}[1]{\ensuremath{\begin{pmatrix}#1\end{pmatrix}}}
\newcommand{\mydet}[1]{\ensuremath{\begin{vmatrix}#1\end{vmatrix}}}
\newcommand{\myaugvec}[2]{\ensuremath{\begin{amatrix}{#1}#2\end{amatrix}}}
\providecommand{\rank}{\text{rank}}
\providecommand{\pr}[1]{\ensuremath{\Pr\left(#1\right)}}
\providecommand{\qfunc}[1]{\ensuremath{Q\left(#1\right)}}
	\newcommand*{\permcomb}[4][0mu]{{{}^{#3}\mkern#1#2_{#4}}}
\newcommand*{\perm}[1][-3mu]{\permcomb[#1]{P}}
\newcommand*{\comb}[1][-1mu]{\permcomb[#1]{C}}
\providecommand{\qfunc}[1]{\ensuremath{Q\left(#1\right)}}
\providecommand{\gauss}[2]{\mathcal{N}\ensuremath{\left(#1,#2\right)}}
\providecommand{\diff}[2]{\ensuremath{\frac{d{#1}}{d{#2}}}}
\providecommand{\myceil}[1]{\left \lceil #1 \right \rceil }
\newcommand\figref{Fig.~\ref}
\newcommand\tabref{Table~\ref}
\newcommand{\sinc}{\,\text{sinc}\,}
\newcommand{\rect}{\,\text{rect}\,}
%%
%	%\newcommand{\solution}[2]{\textbf{Solution:}{#1}}
%\newcommand{\solution}{\noindent \textbf{Solution: }}
%\newcommand{\cosec}{\,\text{cosec}\,}
%\numberwithin{equation}{section}
%\numberwithin{equation}{subsection}
%\numberwithin{problem}{section}
%\numberwithin{definition}{section}
%\makeatletter
%\@addtoreset{figure}{problem}
%\makeatother

%\let\StandardTheFigure\thefigure
\let\vec\mathbf

\bibliographystyle{IEEEtran}
\title{
%	\logo{
Assignment
%	}
}
\author{ Karthikeya hanu prakash kanithi (EE22BTECH11026)}
\maketitle
\parindent0px
\vspace{3cm}
Question : If two events are independent, then
\begin{enumerate}
\item they must be mutually exclusive
\item the sum of their probabilities must be equal to 1
\item (A) and (B) both are correct
\item None of the above is correct
\end{enumerate}
\fi
\solution 
Let $X,Y$ be bernoulli random variables as defined in  \autoref{tab:exemplar/12/13/3/70},
\begin{table}[h]
	\centering
	\input{exemplar/12/13/3/70/tables/Table2.tex}
	\caption{Random variable $X$ declaration}
        \label{tab:exemplar/12/13/3/70}
\end{table}
Lets us consider the pmf's as follows:
\begin{align}
p_X(k) = 
\begin{cases}
  x & \text{if } k = 1 \\
  1-x & \text{if } k = 0 \\
  0 & \text{otherwise}
\end{cases} \\
p_Y(k) = 
\begin{cases}
  y & \text{if } k = 1 \\
  1-y & \text{if } k = 0 \\
  0 & \text{otherwise}
\end{cases}
\end{align}
Two events A and B are independent if and only if:
\begin{align}
\pr{X=1\,,\,Y=1}= p_X(1)p_Y(1) = xy \label{eq:12/13/3/70}
\end{align}
\begin{enumerate}
\item Two events A and B are mutually-exclusive if and only if:
\begin{align}
\pr{X=1\,,\,Y=1}&= 0 \label{eq:12/13/3/70/2}
\end{align}
from \eqref{eq:12/13/3/70} and \eqref{eq:12/13/3/70/2}, We can say that X and Y are not mutually exclusive as
\begin{align}
\pr{X=1\,,\,Y=1} \neq 0 
\end{align}
Hence Option A is false.
\item The sum of the probabilities of two events A and B can be represented by:
\begin{align}
\pr{X=1}+\pr{Y=1} = x+y
\end{align}
Here, It is not always true that $x+y$ should be 1 rather 
\begin{align}
0 \le x+y \le 2 
\end{align}
Hence Option B is false.
\item Option C cant be true as Both A and B are false.
\item So option D none of these is the most appropriate answer 
\end{enumerate}
For example: Let event X and Y be indepedent and defined as follows:
\begin{table}[h]
	\centering
	\input{exemplar/12/13/3/70/tables/Table3.tex}
	\caption{Random variable $X$ declaration}
\end{table}
Then the pmf's as follows:
\begin{align}
p_X(k) = 
\begin{cases}
  \frac{1}{2} & \text{if } k = 1 \\
  \frac{1}{2} & \text{if } k = 0 \\
  0 & \text{otherwise}
\end{cases} \\
p_Y(k) = 
\begin{cases}
  \frac{1}{6} & \text{if } k = 1 \\
  \frac{5}{6} & \text{if } k = 0 \\
  0 & \text{otherwise}
\end{cases}
\end{align}
Then
\begin{enumerate}
\item \begin{align}
\pr{X=1\,,\,Y=1} &= p_X(1)p_Y(1) \\
&= \frac{1}{12} \neq 0 \label{eq:12/13/3/70/3}
\end{align}
From \eqref{eq:12/13/3/70/3}, we can say that X and Y are not mutually exclusive and 
\item \begin{align}
\pr{X=1}+\pr{Y=1} = \frac{2}{3} \label{eq:12/13/3/70/4}
\end{align}
From \eqref{eq:12/13/3/70/4}, the sum of their probabilities may not be equal to 1
\end{enumerate}
Therefore, Option A and B are wrong and Option D is correct





















