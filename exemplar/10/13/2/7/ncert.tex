\iffalse
\let\negmedspace\undefined
\let\negthickspace\undefined
\documentclass[journal,12pt,twocolumn]{IEEEtran}
\usepackage{cite}
\usepackage{amsmath,amssymb,amsfonts,amsthm}
\usepackage{algorithmic}
\usepackage{graphicx}
\usepackage{textcomp}
\usepackage{xcolor}
\usepackage{txfonts}
\usepackage{listings}
\usepackage{enumitem}
\usepackage{mathtools}
\usepackage{gensymb}
\usepackage{comment}
\usepackage[breaklinks=true]{hyperref}
\usepackage{tkz-euclide} 
\usepackage{listings}
\usepackage{gvv}                                        
\def\inputGnumericTable{}                                 
\usepackage[latin1]{inputenc}                                
\usepackage{color}                                            
\usepackage{array}                                            
\usepackage{longtable}                                       
\usepackage{calc}                                             
\usepackage{multirow}                                         
\usepackage{hhline}                                           
\usepackage{ifthen}                                           
\usepackage{lscape}

\newtheorem{theorem}{Theorem}[section]
\newtheorem{problem}{Problem}
\newtheorem{proposition}{Proposition}[section]
\newtheorem{lemma}{Lemma}[section]
\newtheorem{corollary}[theorem]{Corollary}
\newtheorem{example}{Example}[section]
\newtheorem{definition}[problem]{Definition}
\newcommand{\BEQA}{\begin{eqnarray}}
\newcommand{\EEQA}{\end{eqnarray}}
\newcommand{\define}{\stackrel{\triangle}{=}}
\theoremstyle{remark}
\newtheorem{rem}{Remark}
\begin{document}

\bibliographystyle{IEEEtran}
\vspace{3cm}

\title{Exemplar - 10.13.2.7}
\author{EE22BTECH11039 - Pandrangi Aditya Sriram$^{*}$% <-this % stops a space
}
\maketitle
\newpage
\bigskip

\renewcommand{\thefigure}{\theenumi}
\renewcommand{\thetable}{\theenumi}

Apoorv throws two dice once and computes the product of the numbers appearing
on the dice. Peehu throws one die and squares the number that appears on it. Who
has the better chance of getting the number 36? Why?\\\solution
\fi
Let the random variables be defined as:
\begin{table}[!ht]
	\input{exemplar/10/13/2/7/tables/randomvar.tex}
\end{table}
\begin{enumerate}
\item \textbf{Product:} Assuming all dice rolls and equally likely,:
    \begin{align}
        p_X(k) &= 
        \begin{cases}
            \frac{1}{6} & \text{if }k \in \{1, 2, 3, 4, 5, 6\}\\
            0 & \text{otherwise}
        \end{cases}\\
        p_Y(k) &=
        \begin{cases}
            \frac{1}{6} & \text{if }k \in \{1, 2, 3, 4, 5, 6\}\\
            0 & \text{otherwise}
        \end{cases}
    \end{align}
    The probability mass function for Apoorv is:
    \begin{align}
        p_{XY}(k) &= \pr{XY = k}\\
        &= \pr{X = \frac{k}{Y}}\\
        &= E\brak{p_X\brak{\frac{k}{Y}}}\\
        &= \sum_{i = 1}^{6} p_X\brak{\frac{k}{i}} p_Y(i)\\
        &= \frac{1}{6} \sum_{i = 1}^{6} p_X\brak{\frac{k}{i}}\\
        &= \frac{1}{6} \sum_{i = 1}^{6} \frac{\sbrak{k \text{ mod } i = 0}}{6}\sbrak{\frac{k}{i} \leq 6}\\
        &= \frac{1}{36} \sum_{i = 1}^{6} \sbrak{k \text{ mod } i = 0}\sbrak{\frac{k}{i} \leq 6}
    \end{align}
    \begin{figure}[h!]
        \centering
	\includegraphics[width=0.7\columnwidth]{exemplar/10/13/2/7/plots/Apoorv_PMF.png}
        \caption{Sketch of Probability Mass Function for Product obtained by taking a sample of random variables}
    \end{figure}
    Thus, the probability of Apoorv rolling a 36 is:
    \begin{align}
        p_{XY}(36) &= \frac{1}{36} \sum_{i = 1}^{6} \sbrak{36 \text{ mod } i = 0}\sbrak{\frac{36}{i} \leq 6}\\
        &= \frac{1}{36}\brak{0 + 0 + 0 + 0 + 0 + 1}\\
        &= \frac{1}{36} \label{eq:1013273}
    \end{align}
    The cumulative distribution function for Apoorv is:
    \begin{align}
        F_{XY}(k) &= \pr{XY \leq k}\\
        &= \frac{1}{6} \sum_{j = 1}^{k} \sum_{i = 1}^{6} p_X\brak{\frac{j}{i}}\\
        &= \frac{1}{36} \sum_{j = 1}^{k} \sum_{i = 1}^{6}\sbrak{j \text{ mod } i = 0}\sbrak{\frac{j}{i} \leq 6}
    \end{align}
    \begin{figure}[h!]
    	\centering
        \includegraphics[width=0.7\columnwidth]{exemplar/10/13/2/7/plots/Apoorv_CDF.png}
        \caption{Sketch of Cumulative Distribution Function for product obtained by taking a sample of random variables}
    \end{figure}
\item \textbf{Square:}
    The probability mass function for Peehu is:
    \begin{align}
        p_{Z^2}(k) = 
        \begin{cases}
            \frac{1}{6} & \text{if }k \in \{1, 4, 9, 16, 25, 36\}\\
            0 & \text{otherwise}
        \end{cases}
    \end{align}
    Thus, the probability of Peehu rolling a 36 is
    \begin{align}
        p_{Z^2}(36) = \frac{1}{6} \label{eq:1013272}
    \end{align}
    \begin{figure}[h!]
    	\centering
        \includegraphics[width=0.7\columnwidth]{exemplar/10/13/2/7/plots/Peehu_PMF.png}
        \caption{Sketch of Probability Mass Function for square obtained by taking a sample of random variables}
    \end{figure}
    The cumulative distribution function for Peehu is:
    \begin{align}
        F_{Z^2}(k) &= \pr{Z^2 \leq k}\\
        &= 
        \begin{cases}
            0 & \text{if }k \leq 0\\
            \frac{\lfloor\sqrt{k}\rfloor}{6} & \text{if }k \in \{1, 2, ..., 35\}\\
            1 & \text{if }k \geq 36
        \end{cases}
    \end{align}
    \begin{figure}[h!]
    	\centering
        \includegraphics[width=0.7\columnwidth]{exemplar/10/13/2/7/plots/Peehu_CDF.png}
        \caption{Sketch of Cumulative Distribution Function for square obtained by taking a sample of random variables}
    \end{figure}
\end{enumerate}
From \eqref{eq:1013273} and \eqref{eq:1013272}, $p_{Z^2}(36) > p_{XY}(36)$. Therefore, Peehu has a better chance of getting the number 36 than Apoorv.
