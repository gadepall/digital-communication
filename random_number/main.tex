\iffalse
\documentclass[12pt]{article}
\usepackage{graphicx}
\usepackage[none]{hyphenat}
\usepackage[english]{babel}
\usepackage{caption}
\usepackage[parfill]{parskip}
\usepackage{hyperref}
\usepackage{import}
\usepackage{booktabs}
\usepackage{circuit}%%Importing Packing circiut.sty created for circuit
\def\inputGnumericTable{}
\usepackage{color}                                            
    \usepackage{array}                                            
    \usepackage{longtable}                                        
    \usepackage{calc}                                             
    \usepackage{multirow}                                         
    \usepackage{hhline}                                           
    \usepackage{ifthen}
\usepackage{array}
\usepackage{amsmath}  
\usepackage{circuitikz}
\usepackage{parallel,enumitem}
\usepackage{listings}
\lstset{
language=tex,
frame=single,
breaklines=true
}
 
\begin{document}
\fi
%\subsection{Random Number Genration using Shift Registers}
\subsection{Components}
%\begin{table}[!h]
%\centering
\input{random_number/figs/components.tex}

\renewcommand{\theequation}{\theenumi}
\renewcommand{\thefigure}{\theenumi}
\begin{enumerate}[label=\thesubsection.\arabic*.,ref=\thesubsection.\theenumi]
\numberwithin{equation}{enumi}
\numberwithin{figure}{enumi}
\numberwithin{table}{enumi}
\item Generate the CLOCK signal using the 555 timer circuit as shown in the figure \ref{fig:1}
\begin{figure}[!ht]
\includegraphics[width=7cm, height=6cm]{random_number/figs/555.png}
\label{fig:1}
\end{figure}	
	
\item Connect the CLOCK output of 555 timer circuit to CLOCK signal of D-Flip flops, change the resistor value to 1M$\Omega$		
\item Now make the cicuit for shift registers uisng 4 D-Flip flops (by using two 7474 IC's) and one X-OR gate (7486 IC) as shown in figure \ref{fig:2}. Pin out for 7474 IC is shown in figure \ref{fig:7447}
\begin{figure}
\centering
\subimport{random_number/figs/}
{circuit1.tex}
\caption{Circuit Connections}
\label{fig:2}
\end{figure}

\begin{figure}[!ht]
\begin{center}

\includegraphics[width=11cm, height=6cm]{random_number/figs/IC7474.png}
\label{fig:7474}

\end{center}
\end{figure}
\item Connect the output of each D-flip flop to Decoder IC (7447 IC), The pin out of 7447 IC is shown in figure \ref{fig:7447}
\begin{figure}[!ht]
\begin{center}
\resizebox {\columnwidth} {!} {
\input{random_number/figs/7447.tex}
}
\end{center}
\caption{}
\label{fig:7447}
\end{figure}
\item As per the pinout of IC 7474 [2,12] pins of both IC's need to connected to the [7,1,2,6] of decoder IC respectively
\item Make connections between the seven segment display in Fig\ref{fig:sevenseg} and the 7447 IC in Fig.\ref{table:7447_disp} as shown in Table 2.1
\begin{table}[!ht]
\centering
\input{random_number/figs/7447_disp.tex}
\caption{}
\label{table:7447_disp}
\end{table}
\begin{figure}[!ht]
\begin{center}
\resizebox {0.3\columnwidth} {!} {
\input{random_number/figs/sevenseg.tex}
}
\end{center}
\caption{}
\label{fig:sevenseg}
\end{figure}
\item Additionally make conections like Vcc and GNG to every IC as per the respective IC pinout for IC's 7474,7447,7486.
\end{enumerate}
